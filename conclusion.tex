\section{Conclusions and open research areas}
\label{sec:trends}

This research is the first to review and classify the MTVRPTW literature.  In this study, we develop path-based flow formulations to introduce the MTVRPTW and its variants.  Then, we offer a taxonomic review of the MTVRPTW literature, including applications, mathematical models, and solution algorithms.  \add{This section also offers our perspective on current research trends and identifies areas of potential research focus.}
\newline

%In this section, we analyze trends in the MTVRPTW literature concerning the following aspects: (TODO: change this) choice of the MTVRPTW feature (i.e., which features are best suited for specific applications); exact and heuristic algorithm designs.  We also propose possible research areas, where applicable.
%\newline

% TODO: why some variants are more important (practically)? What is their practical importance?
% \newline

\add{Recent studies on the MTVRPTW have focused on real-life problems that involve various real-world aspects.  These complex real-life MTVRPTWs involve more complex constraints and objectives.  That is, real-life and rich MTVRPTWs appear to be the trend of the research on MTVRPTW.  According to \cite{lahyani2015rich}, a Rich Vehicle Routing Problem (RVRP) must extend the standard VRP by considering not less than four strategic or tactical characteristics regarding the distribution system and containing no less than six daily restrictions related to the physical characteristics.  In addition, if a VRP is mostly defined by strategic and tactical aspects (resp. by physical characteristics), at least five of them (resp. at least nine of them) must appear in an RVRP.  Based on their definition of RVRPs, some variants of the MTVRPTW in the literature may not be considered as rich.  Nonetheless, these previous studies seek to consider complex aspects of reality and show the trend toward real-life MTVRPTWs.}
\newline

% To capture various aspects of real-world vehicle routing problems, studies in the literature usually consider multiple features simultaneously (i.e., rich MTVRPTW variants).  For example, we have found that the following features are frequently grouped together: profit (P), service-dependent loading time (SDLT), and limited trip duration (LTD).  These features are particularly popular in studies with exact solution algorithms.  Moreover, the majority of exact algorithms in the literature rely on the enumeration of non-dominated paths, which is not suitable for MTVRPTW variants with loose deadline constraints (i.e., trip duration).  Furthermore, multiple mathematical formulations exist for the same MTVRPTW variant (e.g., MTVRPTPW-SDLT-LTD).  In our opinion, determining the most effective formulation for popular MTVRPTW variants deserves consideration in future research.  \add{We also note that there are real-world constraints that are not well-studied in the MTVRPTW literature.  For instance, \cite{anaya2016biomedical} consider multiple pick-up and delivery of perishable biomedical samples, which adds complexity to the formulation and solution.}

% We thus believe that research on new enumeration methods is needed to tackle such variants effectively.  Another potential direction is to extend the work in \cite{neira2020new}, which propose new MILP formulations capable of solving an MTVRPTW variant without trip duration limit (i.e., MTVRPTW-SDLT).

% citation for papers require non-dominated: (e.g., Azi et al., 2010 \cite{azi2010exact}; Macedo et al., 2011 \cite{macedo2011solving}; Macedo et al., 2012 \cite{macedo2012generalized}; Hernandez et al., 2014 \cite{hernandez2014new}; Hernandez et al., 2016 \cite{hernandez2016branch}; Paradiso et al., 2020 \cite{paradiso2020exact})

% For heuristic algorithms, the initial step of each algorithm can be generalized as a routing phase, which produces either a feasible solution or a candidate set of trips.  This first phase is relatively independent of the following phase(s), meaning that we can plug in different procedures for routing, but the algorithm still generates a final feasible solution.  However, we believe there is a lack of focus in the literature on analyzing the impact of the routing phase on the overall MTVRPTW algorithm's performance.  In other words, whether we should use a specific heuristic (in the routing phase) suitable for the corresponding metaheuristic or not (i.e., different routing heuristics have minimal impact on the final result) is an open question worth exploring in future research.  

\add{Concerning solution algorithms, as indicated by \cite{lahyani2015rich}, exact algorithms are rarely proposed to solve RVRPs because of their limited capability to tackle large-scale instances of such complex problems.  Likewise, exact algorithms are not, at least currently, promising for tackling various large-scale MTVRPTWs.  Indeed, existing studies of exact solution algorithms in the MTVRPTW literature are primarily confined to a few features: Profit, Dependent Loading or Unloading Times, and Limited Trip Duration.  In general, the most popular methods for solving real-life and rich MTVRPTWs are metaheuristics, namely: ILS, ALNS, SA, Memetic Algorithm, and adaptive memory procedure (AMP).  We note that metaheuristics in the MTVRPTW literature usually incorporate procedures customized for specific variants, which complicates the comparative study of different metaheuristics for MTVRPTWs.  Furthermore, hybrid solution techniques (i.e., matheuristics) that combine metaheuristics and exact methods have been successfully utilized to tackle large instances of complex RVRPs within a reasonable time \citep{doerner2010survey, goel2020hybrid}.  Thus, we believe that matheuristics are promising approaches for solving large-scale MTVRPTWs.}
\newline

% While it is possible to adapt an algorithm designed for one variant to solve another, we find such benchmarking method challenging as authors often include procedures customized for specific variants within their metaheuristics, leading to an unbalanced comparison. 
% Regardless of the choice of metaheuristic, we found that it is common to incorporate both time window features, such as critical time intervals, and LS moves tailored to multi-trip in MTVRPTW heuristic algorithms.

% Various heuristics (or metaheuristics) have been proposed to solve the MTVRPTW and its variants: ILS, ALNS, SA, Memetic Algorithm, and adaptive memory procedure (AMP).  A balance between diversification (i.e., exploration of new solution elements) and intensification (i.e., exploitation of champion features) is essential for efficient metaheuristics \citep{vidal2013heuristics}, albeit often paid little attention in the MTVRPTW literature.  Therefore, we believe that parameter analysis (i.e., tuning) for MTVRPTW solution algorithms is a promising future research area.  

% Subsequent research can also focus on conducting comparative studies of various heuristic algorithms for the MTVRPTW.  While it is possible to adapt an algorithm designed for one variant to solve another, we find such benchmarking method challenging as authors often include procedures customized for specific variants within their metaheuristics, leading to an unbalanced comparison.  Regardless of the choice of metaheuristic, we have found that it is common to incorporate both time window features, such as critical time intervals, and LS moves tailored to multi-trip in MTVRPTW heuristic algorithms.
% \newline

% Regardless of the choice of metaheuristic, we believe that emphases on both time window features, such as critical time intervals, and LS moves tailored to multi-trip, are essential in designing state-of-the-art heuristic algorithms for the MTVRPTW.

\add{Beyond the aforementioned methods, \textit{Reinforcement Learning} (RL) techniques also show potential in tackling MTVRPTWs, building on their successful application in solving VRPs.  For example, \cite{zhang2020multi} propose a multi-agent RL model for the VRP with soft time windows that outperforms classical heuristics in terms of computation time; \cite{lu2020learning} combine RL and classical heuristics to address the VRP by using RL-based controller to select the improvement operator in an iterative algorithm.  \cite{raza2022vehicle}, which survey recent advancements in solving VRPs using RL, note that there is a lack of RL studies focusing on real-life constraints.  To our knowledge, RL studies targeting the MTVRPTW and its variants have not yet been proposed.  Moreover, as RL models that work well for a problem may perform inferior for other variants \citep[see,][]{li2021deep}, we believe that it is worth researching RL frameworks with a high level of flexibility to solve different MTVRPTW variants effectively.}

% Apart from the heuristics mentioned above, \textit{Reinforcement Learning} (RL) techniques have been applied to solve VRPs \citep[see,][]{raza2022vehicle}.  \cite{raza2022vehicle} also point out a lack of RL studies focusing on real-life VRP constraints.  To our knowledge, RL models designed for both the time windows and multi-trip characteristics of the VRP (i.e., the MTVRPTW), have not yet been proposed.  Nevertheless, deep RL and multi-agent RL have shown their potential to tackle the VRP with time windows \citep[see,][]{gupta2022deep}, the VRP with soft time windows \citep[see,][]{zhang2020multi}, and the VRP with heterogeneous fleet \citep[see,][]{li2021deep}
%https://pubsonline.informs.org/doi/abs/10.1287/inte.2021.1108, also check \cite{raza2022vehicle}).  

% Although generalized RL frameworks exist for the general planning problems \citep[see,][]{groshev2018learning}, problem-specific policies utilizing both time window and multi-trip characteristics of the MTVRPTW can yield better results in terms of both solution quality and runtime.

% However, we found a lack of RL studies focusing on real-life MTVRPTWs.  Furthermore, a major limitation of RL algorithms is the need for re-training for new problem variants \citep{raza2022vehicle}.

%Thus, we believe applying RL to handle real-life and rich MTVRPTWs can be a promising direction for future study.
% However, \cite{raza2022vehicle} also point out a lack of RL studies focusing on real-life VRP constraints.

% Apart from the methods mentioned above, reinforcement learning has shown its potential to tackle hard combinatorial optimization problems.  \cite{bello2016neural} propose a framework based on neural networks and reinforcement learning, which can find near-optimal solutions for the Travelling Salesman Problem instances with up to 100 customers.  For the VRP, \cite{nazari2018reinforcement} present a reinforcement learning framework that outperforms classical heuristics on medium-sized instances.  Recently, \cite{gupta2022deep} suggest a deep Q-network method to solve large-scale instances of the VRPTW.  To our knowledge, reinforcement learning and machine learning, in general, have not yet been used to solve the MTVRPTW.  Hence, applying machine learning to handle the MTVRPTW and its variants can be a promising direction for future study.


%TODO: which variant lacks attention in the literature (for exact / heuristic method); which feature is not being directly attacked (like algorithms not designed with that feature in mind) -- "due to the large number of constraints in MTVRPTW, some features like abc xyz are often left un-attacked (i.e., algorithms not tailored for such feature)"
