\section{Introduction}

% The Vehicle Routing Problem (VRP) aims to assign routes to a vehicle fleet to serve a set of customers.
The Vehicle Routing Problem (VRP) has strong applications across numerous industries, such as transportation, logistics, communications, manufacturing, and military  \citep{vidal2013heuristics}.  To address various practical applications, many extensions have been explored: the VRP with Time Windows (VRPTW), the Multi-Trip VRP (MTVRP), the VRP with Pickup and Delivery (VRPPD), the Multi-Depot VRP (MDVRP), the Heterogeneous VRP (HVRP), the Time-Dependent VRP (TDVRP), the Periodic VRP (PVRP), etc.  Several previous studies have been conducted to classify these variants.  \cite{caceres2014rich} review more than 50 papers dealing with the \textit{rich} VRP trend, where multiple constraints are combined to model realistic problems. Another comprehensive review of the VRP articles between 2009 and 2015 is given by \cite{braekers2016vehicle}, in which the authors categorized articles and analyzed trends in the VRP literature.  \cite{cattaruzza2016vehicle} survey the Multi-Trip VRP (MTVRP) extensions, in which vehicles are permitted to make more than one trip, together with the exact and heuristic approaches to tackle them.  They identify the MTVRPTW as an extension of the MTVRP.  Recently, \cite{elshaer2020taxonomic} analyzed 299 articles in the literature from 2009 to 2017, classified metaheuristic algorithms for the VRP, and gave an exhaustive list of the VRP variants.  \cite{vidal2020concise} offer a summary of both current and emerging VRP variations.  \add{\cite{mor2022vehicle} survey the recent development in MTVRP research following \cite{cattaruzza2016vehicle}.}  Lastly, it is worthwhile to mention the two well-known books edited by \cite{toth2002vehicle, toth2014vehicle} that are devoted to the VRP and its variants.
\newline

The MTVRP was initially introduced by \cite{fleischmann1990vehicle}, while the VRPTW has been studied by \cite{kolen1987vehicle}. Since then, time windows and multi-trip have been among the most widely-studied features, enforcing a time window at a customer that the service must be performed while allowing the vehicles to make multiple trips.  In terms of practical application, both time window and multi-trip allowance are crucial features in the field of urban logistics, which includes urban goods distribution, city logistics, and last mile logistics \citep{cardenas2017city}.  Customers in urban logistics, especially in last-mile logistics, usually if not always specify their service time windows.  Moreover, due to the close proximity of customers in urban areas, drivers (vehicles) must make multiple trips during their hours of work by small-capacity vehicles.  \addb{Not surprisingly, multi-trip routing problems with time windows have also been applied to fields other than urban logistics \citep[see, e.g.,][]{christiansen2017operational, liu2018branch, tang2015exact}.}  Apart from the rich real-world applications, the MTVRPTW also requires quite different problem modeling and solution algorithms from those of the VRP and other variants.  Hence, it is worthwhile and important to fully review the MTVRPTW and its extensions.  Although research on the MTVRPTW has gained significant focus lately, the definition and naming for the MTVRPTW vary widely in the literature, and no research has attempted to conduct a taxonomic classification of the problem.  Therefore, this research aims to fill this gap by formally defining the problem and introducing its variants together with solution algorithms to tackle them.
\newline

%There are two different objectives frequently considered in the MTRVPTW literature: optimizing travel cost and minimizing fleet size.  The former is common in settings where the variable routing cost makes up most of the total cost.  For example, a logistics company aims to minimize operational costs by designing daily driver schedules for a fixed fleet of vehicles.  The latter is of practical relevance for applications in which the fixed investment cost of vehicles is more significant compared to the daily operating expense, for instance, an e-commerce platform using leased vehicles for its last-mile delivery, where the pricing model is per-vehicle.  In practice, most papers in the MTVRPTW literature regard travel cost optimization as the primary objective.  Furthermore, it is more straightforward to extend this objective to formalize other MTVRPTW variants.  Thus, this paper will use travel cost as the objective in the basic model of the MTVRPTW.  The fleet size minimization objective will be reviewed subsequently in an MTVRPTW variant (Section \ref{subsec:minimum}).
%\newline

%Although the multi-trip aspect of the MTVRPTW may imply an objective function of minimizing fleet size (i.e., allowing multiple trips to use fewer vehicles), we consider optimizing travel distance as the primary objective (see Section \ref{sec:formulation}).  That is because in the city logistics and last-mile delivery contexts, the cost per vehicle can be significant, but usually a fixed investment, while the operating expense is largely contributed by variable routing cost proportional to the travel distance.  For example, a logistics company aims to minimize operational costs by designing daily driver schedules for a fixed fleet of vehicles.  In such applications, it is not beneficial to travel a longer distance to use less than a pre-defined number of vehicles.  Nevertheless, we believe that the most appropriate objective function depends on the actual situation. Many applications would benefit from the smallest number of vehicles (e.g., an e-commerce platform using leased vehicles to deliver goods where the pricing model is per-vehicle).  Thus, we also explore the fleet size minimization objective function in an MTVRPTW variant (Section \ref{subsec:minimum}).
%\newline

\add{We organize this survey paper into six sections.}
%Rest of this paper is organized as follows.  
Section \ref{sec:formulation} develops a mathematical formulation for the MTVRPTW and introduces popular test instances.  Section \ref{sec:variants} classifies the MTVRPTW variants based on the constraints and objectives.  \add{Sections \ref{sec:exact} and \ref{sec:heuristics} review existing exact and heuristics algorithms for the MTVRPTWs respectively.}% while heuristic algorithms are presented in Section \ref{sec:heuristics}.  
\add{Finally,} section \ref{sec:trends} \add{concludes the study} and suggests promising areas for future research.
