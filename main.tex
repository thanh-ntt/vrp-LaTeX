%\documentclass{cas-sc}
\documentclass[11pt]{article}
\usepackage{times}
\usepackage{natbib}
\usepackage{fullpage}
\let\printorcid\relax % Remove ORCID footnote

\usepackage [english]{babel}
\usepackage [autostyle, english = american]{csquotes}
\MakeOuterQuote{"}

\bibliographystyle{apa}
\AtBeginDocument{\renewcommand{\harvardand}{and}}
\setcitestyle{authoryear,open={(},close={)}} %Citation-related commands

\usepackage[utf8]{inputenc}
\usepackage{mathtools}
\usepackage{multirow}
\usepackage{longtable}
\usepackage{colortbl}
\usepackage{ragged2e}
\usepackage{xcolor}
\usepackage{booktabs}
\usepackage{tablefootnote}
\usepackage{lscape}
\usepackage{float}
\usepackage{url}
\usepackage{threeparttable}
\usepackage[onehalfspacing]{setspace}
\floatstyle{plaintop}
\restylefloat{table}

\linespread{1.5}

\newcommand\thefontsize{\expandafter\string\the\font}
%\newcommand\add[1]{\textcolor{blue}{#1}}
\newcommand\add[1]{\textcolor{black}{#1}}
\newcommand\addb[1]{\textcolor{black}{#1}}


% \usepackage[onehalfspacing]{setspace}

%\newcommand{\onehalfspacing}{%
%  \setstretch {1.2135}
%}

%%%%%%%%%%%%%%%%%%%%%%%%%%%%%%%%%%%%%%%%%%%%%%%%%%%%%%%%
\title{Multi-Trip Vehicle Routing Problem with Time Windows: A Classification and Review}
\begin{document}

%\title[mode=title]{Multi-Trip Vehicle Routing Problem with Time Windows: A Classification and Review}
%\shorttitle{}
\maketitle

%%%%%%%%%%%%%%%%%%%%%%%%%%%%%%%%%%%%%%%%%%%%%%%%%%%%%%%%

\begin{abstract}
The Vehicle Routing Problem (VRP) is extensively studied in combinatorial optimization and has diverse practical applications. The recent trend extends the VRP with real-life constraints.  Among them, two popular constraints are (i) time window: the service of each customer must be performed within a given time interval; and (ii) multiple trips: a vehicle may perform more than one trip.  A combination of the two features leads to the Multi-Trip Vehicle Routing Problem with Time Windows (MTVRPTW). The MTVRPTW and its extensions have been studied actively in the past few years.  While the problem constraints and objectives vary greatly among different articles, a comprehensive review of MTVRPTW literature is missing.  Therefore, this research aims to conduct such a review of previous studies published between 2009 and 2023.  In this review, we define the MTVRPTW by giving a mathematical formulation; survey the variants of the MTVRPTW; review the exact and heuristic solution algorithms; and suggest possible future research directions.
\end{abstract}

%\begin{keywords}
\textbf{Keywords:} Vehicle routing problem;
Time window; 
Multiple trips;
City logistics;
Review
\maketitle

\section{Introduction}

% The Vehicle Routing Problem (VRP) aims to assign routes to a vehicle fleet to serve a set of customers.
The Vehicle Routing Problem (VRP) has strong applications across numerous industries, such as transportation, logistics, communications, manufacturing, and military  \citep{vidal2013heuristics}.  To address various practical applications, many extensions have been explored: the VRP with Time Windows (VRPTW), the Multi-Trip VRP (MTVRP), the VRP with Pickup and Delivery (VRPPD), the Multi-Depot VRP (MDVRP), the Heterogeneous VRP (HVRP), the Time-Dependent VRP (TDVRP), the Periodic VRP (PVRP), etc.  Several previous studies have been conducted to classify these variants.  \cite{caceres2014rich} review more than 50 papers dealing with the \textit{rich} VRP trend, where multiple constraints are combined to model realistic problems. Another comprehensive review of the VRP articles between 2009 and 2015 is given by \cite{braekers2016vehicle}, in which the authors categorized articles and analyzed trends in the VRP literature.  \cite{cattaruzza2016vehicle} survey the Multi-Trip VRP (MTVRP) extensions, in which vehicles are permitted to make more than one trip, together with the exact and heuristic approaches to tackle them.  They identify the MTVRPTW as an extension of the MTVRP.  Recently, \cite{elshaer2020taxonomic} analyzed 299 articles in the literature from 2009 to 2017, classified metaheuristic algorithms for the VRP, and gave an exhaustive list of the VRP variants.  \cite{vidal2020concise} offer a summary of both current and emerging VRP variations.  \add{\cite{mor2022vehicle} survey the recent development in MTVRP research following \cite{cattaruzza2016vehicle}.}  Lastly, it is worthwhile to mention the two well-known books edited by \cite{toth2002vehicle, toth2014vehicle} that are devoted to the VRP and its variants.
\newline

The MTVRP was initially introduced by \cite{fleischmann1990vehicle}, while the VRPTW has been studied by \cite{kolen1987vehicle}. Since then, time windows and multi-trip have been among the most widely-studied features, enforcing a time window at a customer that the service must be performed while allowing the vehicles to make multiple trips.  In terms of practical application, both time window and multi-trip allowance are crucial features in the field of urban logistics, which includes urban goods distribution, city logistics, and last mile logistics \citep{cardenas2017city}.  Customers in urban logistics, especially in last-mile logistics, usually if not always specify their service time windows.  Moreover, due to the close proximity of customers in urban areas, drivers (vehicles) must make multiple trips during their hours of work by small-capacity vehicles.  \addb{Not surprisingly, multi-trip routing problems with time windows have also been applied to fields other than urban logistics \citep[see, e.g.,][]{christiansen2017operational, liu2018branch, tang2015exact}.}  Apart from the rich real-world applications, the MTVRPTW also requires quite different problem modeling and solution algorithms from those of the VRP and other variants.  Hence, it is worthwhile and important to fully review the MTVRPTW and its extensions.  Although research on the MTVRPTW has gained significant focus lately, the definition and naming for the MTVRPTW vary widely in the literature, and no research has attempted to conduct a taxonomic classification of the problem.  Therefore, this research aims to fill this gap by formally defining the problem and introducing its variants together with solution algorithms to tackle them.
\newline

%There are two different objectives frequently considered in the MTRVPTW literature: optimizing travel cost and minimizing fleet size.  The former is common in settings where the variable routing cost makes up most of the total cost.  For example, a logistics company aims to minimize operational costs by designing daily driver schedules for a fixed fleet of vehicles.  The latter is of practical relevance for applications in which the fixed investment cost of vehicles is more significant compared to the daily operating expense, for instance, an e-commerce platform using leased vehicles for its last-mile delivery, where the pricing model is per-vehicle.  In practice, most papers in the MTVRPTW literature regard travel cost optimization as the primary objective.  Furthermore, it is more straightforward to extend this objective to formalize other MTVRPTW variants.  Thus, this paper will use travel cost as the objective in the basic model of the MTVRPTW.  The fleet size minimization objective will be reviewed subsequently in an MTVRPTW variant (Section \ref{subsec:minimum}).
%\newline

%Although the multi-trip aspect of the MTVRPTW may imply an objective function of minimizing fleet size (i.e., allowing multiple trips to use fewer vehicles), we consider optimizing travel distance as the primary objective (see Section \ref{sec:formulation}).  That is because in the city logistics and last-mile delivery contexts, the cost per vehicle can be significant, but usually a fixed investment, while the operating expense is largely contributed by variable routing cost proportional to the travel distance.  For example, a logistics company aims to minimize operational costs by designing daily driver schedules for a fixed fleet of vehicles.  In such applications, it is not beneficial to travel a longer distance to use less than a pre-defined number of vehicles.  Nevertheless, we believe that the most appropriate objective function depends on the actual situation. Many applications would benefit from the smallest number of vehicles (e.g., an e-commerce platform using leased vehicles to deliver goods where the pricing model is per-vehicle).  Thus, we also explore the fleet size minimization objective function in an MTVRPTW variant (Section \ref{subsec:minimum}).
%\newline

\add{We organize this survey paper into six sections.}
%Rest of this paper is organized as follows.  
Section \ref{sec:formulation} develops a mathematical formulation for the MTVRPTW and introduces popular test instances.  Section \ref{sec:variants} classifies the MTVRPTW variants based on the constraints and objectives.  \add{Sections \ref{sec:exact} and \ref{sec:heuristics} review existing exact and heuristics algorithms for the MTVRPTWs respectively.}% while heuristic algorithms are presented in Section \ref{sec:heuristics}.  
\add{Finally,} section \ref{sec:trends} \add{concludes the study} and suggests promising areas for future research.

\section{Problem description, formulation, and test instances}
\label{sec:formulation}

To address the MTVRPTW, two objectives are commonly considered in the MTVRPTW literature: 1) minimizing total route cost; and 2) minimizing the number of vehicles used plus the route cost associated with these vehicles.  The former is commonly used in the setting where all the vehicles must be used to perform tasks to ensure, for example, even workload distribution among drivers.  The latter is popular in applications where it is possible to leave some vehicles unused, and thus the number of vehicles used must be determined, in which each used vehicle incurs a fixed cost.  It follows that the objective of the applications requires the minimization of the total fixed costs associated with the vehicles and the total route cost incurred by using the vehicles.  Most papers in the MTVRPTW literature focus on the minimization of total route cost.  This objective is also more suitable for describing the objectives of other MTVRPTW variants.  Thus, this research develops a mathematical programming model of the MTVRPTW with the minimization of total route cost as the objective.  Then, this model is treated as the basis for introducing MTVRPTW variants.
\newline

So far, different formulations of the MTVRPTW have been proposed. For example, \cite{hernandez2014new} introduce an arc-based formulation; \cite{paradiso2020exact} develop a formulation based on structure; and \cite{azi2010exact} and \cite{macedo2011solving} incorporate both path and arc elements in their formulations.  However, these proposed formulations are not suitable for introducing the MTVRPTW and its variants.  Therefore, we develop a path flow formulation to achieve the goal of the review article. 


\subsection{Problem description}
\label{sec:math-formulation}

The MTVRPTW is defined as follows:  Consider $G = (N, A)$ as a directed network comprising a node set $N = V \cup \{0, n + 1\}$ and an arc set $A$.  Here, $V = \{1, 2,..., n\}$ represents the customer nodes, while the depot is indicated by either $0$ or $n + 1$.  Each arc $(i, j) \in A$ has a cost $c_{ij}$ and travel time $t_{ij}$.  Each customer $i \in V$ is associated with a demand $q_i$, a service time $s_i$, and a service time window $[a_i, b_i]$.  The vehicle must reach $i$ before $b_i$, and if it arrives earlier than $a_i$, it must wait until the service begins (i.e., hard time window).  For the depot, $s_0$ models a constant loading time, and $[a_0, b_0] = [a_{n+1}, b_{n+1}] = [E, L]$, in which $E$ represents the earliest possible departure from the depot, and $L$ represents the latest possible arrival at the depot for all trips.  At the depot, there is a homogeneous fleet of $K$ vehicles with fixed capacity of $Q$.  A vehicle begins at the depot, serves some customers, and returns to the depot; these steps constitute a trip, and a vehicle may perform multiple trips.  Some authors use the terms “trip” and “route” interchangeably. Hence, to avoid confusion, we use only “trip” in this paper.  In addition, we adopt the definition from \cite{cattaruzza2016vehicle} and define the multiple trips allocated to a single vehicle as a "journey".  A trip $p$ is represented as an ordered list of nodes $(0, i_{p_1}, i_{p_2}, ..., n + 1)$, and is a feasible elementary path (i.e., satisfying vehicle capacity and time window constraints).  In addition, each trip $p$ is associated with travel cost $c_p = \sum_{(i, j) \in p} c_{ij}$.  The route cost in the MTVRPTW is calculated by adding up the costs associated with the chosen trips in a solution.


%The Multi-Trip Vehicle Routing Problem with Time Windows (MTVRPTW) is defined as follows: we have a set of nodes $V^+ = V \cup \{0, n + 1\}$, where $V = \{1, 2,..., n\}$ is the set of customer nodes and the depot is denoted by $0$ or $n + 1$.  $A$ is the arc set, with each arc $(i, j) \in A$ is associated with a distance $d_{ij}$ and a travel time $t_{ij}$.  Each customer $i \in V$ is characterized by a demand $q_i$, a service time $s_i$, and a time window $[a_i, b_i]$.  The vehicle must arrive at $i$ before $b_i$, and if it arrives before $a_i$, it must wait until the time window begins (i.e., hard time window). For the depot, $s_0$ models a constant loading time, and $b_0 = T$ models a planning horizon for all trips.  A homogeneous fleet of $K$ vehicles with fixed capacity $Q$ is available at the depot.  A vehicle starts from the depot, visits some customers, then comes back to the depot; these steps constitute a trip, and a vehicle may perform multiple trips.  Some authors use the terms "trip" and "route" interchangeably, so to avoid confusion, we will avoid using "route" in this paper to avoid confusion.  We will also refer to the multiple trips assigned to the same vehicle as a "journey" (Cattaruzza et al., 2016b \cite{cattaruzza2016vehicle}).
%\newline

%The Multi-Trip Vehicle Routing Problem with Time Windows (MTVRPTW) is defined as follows: let $G = (V^+, D)$ be a weighted complete graph.  $V^+ = V \cup \{0, n + 1\}$, with $V = \{1, ..., n\}$ represents the set of customer nodes, and $\{0, n + 1\}$ denotes the depot.  $D$ represents the set of arcs, where each arc $(i, j)$ is associated with the routing cost $d_{ij}$ from customer $i$ to customer $j$.  Each customer $i \in V$ is characterized by a demand $q_i$, a service time $s_i$, and a time window $[a_i, b_i]$.  The vehicle must arrive at $i$ before $b_i$, and if the vehicle arrives before $a_i$, it must wait until the time window begins (i.e., hard time window). For the depot, $s_0$ models a constant loading time, and $b_0 = T$ models a planning horizon for all trips.
%\newline

%A trip $p$ can be represented as an ordered list of nodes $(0, i_{p_1}, i_{p_2}, ..., i_{p_{\mu_p}}, n + 1)$, where $\mu_p$ is the number of visited customers.  For any trip $p$, coefficients $t^p_0$, $t^p_{n+1}$ and $t^p_i$ represent the trip starting time (i.e., when the vehicle leaves the depot), the trip ending time (i.e., when the vehicle comes back to the depot), and the service starting time at customer $i$, respectively.  Note that $t^p_i = 0$ if customer $i$ is not served by the trip $p$ and trip duration excludes vehicle loading time.  For each customer $i \in V$, a coefficient $\tau^p_i$ describes the number of times the customer is visited in trip $p$.  For any two customers $i, j \in V$, a binary coefficient $w^p_{ij}$ denotes whether if $i$ and $j$ are visited consecutively by trip $p$ in that order.  $P$ is the set of all feasible trips, where a trip is feasible if and only if it satisfies vehicle load constraints and time window constraints for all customers.  In addition, each trip $p$ is associated with a travel cost $c_p$, which can be the total traveled distance or time of the trip.
%\newline

%A fixed fleet of $K$ identical vehicles with vehicle capacity $Q$ is available at the depot.  A vehicle starts from the depot, visits some customers, then comes back to the depot, these steps constitute a trip.  Each vehicle is allowed to perform multiple trips because, in practice, it is often more cost-effective to reuse an existing vehicle rather than introduce a new one.  Some authors use the terms "trip" and "route" interchangeably, however, to avoid confusion, we will no longer use "route" in this paper.  We shall follow the convention suggested by Cattaruzza et al. (2016b) \cite{cattaruzza2016vehicle}, which refers to a sequence of customer services starting and ending at the depot without an intermediate stop at the depot as a "trip", and refers to the multiple trips assigned to the same vehicle as a "journey".  Additionally, as we model a homogeneous fleet of $K$ vehicles with constant speed, we can conveniently convert between the travel time and distance, thus, using the same unit of measurement (see Azi et al., 2014 \cite{azi2014adaptive}).  Similar to the VRPTW, the routing cost (i.e., routing cost) is defined as the total travel distance, which excludes the waiting time.
%\newline

% The objective of the MTVRPTW is to determine an assignment of vehicle trips to customers that minimizes the number of used vehicles (i.e., fleet size), breaking ties in favor of the minimum routing cost, while satisfying the following conditions:

% The objective of the MTVRPTW is to determine an assignment of vehicle trips to customers that minimizes the total travel cost while satisfying the following conditions:
% \begin{enumerate}
% \item Each customer is visited exactly once within its service time window;
% \item The sum of customers' demands in any trip does not exceed vehicle capacity.
% \end{enumerate}

\subsection{Model formulation}
\label{sec:model}
This research formulates the MTVRPTW as a \textit{mixed integer linear program} (MILP) that is a path flow formulation.  Below, we show the notation used to construct the mathematical model, and then the program itself.

\subsubsection{Notations}
\textbf{Set:}
\newline
$P$: set of all feasible elementary trips from depot $0$ (source) to depot $n + 1$ (sink)
\newline
$V$: set of customers
\newline
% $p_1 < p_2$: trip $p_1 \in P$ and trip $p_2 \in P$ are performed by the same vehicle, and $p_1$ is performed before $p_2$
% \newline

\noindent
\textbf{Parameter:}
\newline
$K$: maximum number of vehicles available
\newline
$Q$: vehicle capacity
\newline
$t_{ij}$: travel time associated with arc $(i, j) \in A$
\newline
$c_p$: travel cost associated with path $p \in P$
\newline
$q_i$: demand associated with customer $i \in V$
\newline
$s_i$: service time associated with customer $i \in V$
\newline
$s_0$: loading time at depot $0$
\newline
$[a_i, b_i]$: time window associated with customer $i \in V$
\newline
$[a_0, b_0] = [a_{n + 1}, b_{n + 1}]$: depot operation time
\newline
$\vartheta^p_i$: customer-trip indicator coefficient such that $\vartheta^p_i = 1$ if trip $p$ passes customer $i$, and $\vartheta^p_i = 0$ otherwise
\newline
$\delta^p_{ij}$: arc-trip indicator coefficient such that $\delta^p_{ij} = 1$ if arc $(i, j) \in A$ appears in trip $p \in P$, and $\delta^p_{ij} = 0$ otherwise
\newline
$M$: a sufficiently large number
\newline

\noindent
\textbf{Variable:}
\newline
$x_p$: $x_p = 1$ if $p \in P$ is a part of the optimal solution, and $x_p = 0$ otherwise
\newline
$y^p_i$: $y^p_i = 1$ if customer $i \in V$ is served in trip $p \in P$, and $y^p_i = 0$ otherwise
\newline
$t^p_i$: service start time at node $i \in N$ of trip $p \in P$
\newline
$z_{{p_1}{p_2}}$: $z_{{p_1}{p_2}} = 1$ if trip $p_1 \in P$ is immediately followed by trip $p_2 (> p_1) \in P$ of the same vehicle, and $z_{{p_1}{p_2}} = 0$ otherwise

\subsubsection{Mathematical model}

\begin{spreadlines}{14pt} % ooiwt - from mathtools
\allowdisplaybreaks
\begin{align} 
Min &\quad \sum_{p \in P} c_p x_p &\quad
\label{eq1} \\
    s.t. \quad y^p_i &= x_p \vartheta^p_i, &\quad \forall i \in V, \quad \forall p \in P,
\label{eq2} \\
    \sum_{p \in P} y^p_i &= 1, &\quad \forall i \in V,
\label{eq3} \\
    \sum_{i \in V} q_i y^p_i &\leq Q, &\quad \forall p \in P,
\label{eq4} \\
    a_i y^p_i \leq t^p_i &\leq b_i y^p_i, &\quad \forall i \in N, \quad \forall p \in P,
\label{eq5} \\
    t^p_i + s_i + t_{ij} &\leq t^p_j + M(1 - \delta^p_{ij} x_p), &\quad \forall i, j \in V, \quad \forall p \in P,
\label{eq6} \\
    t^{p_1}_{n + 1} + s_0 &\leq t^{p_2}_0 + M(1 - z_{{p_1}{p_2}}), &\quad \forall p_1, p_2 \in P, \quad p_1 < p_2,
\label{eq7} \\
    \sum_{p \in P} x_p - \sum_{p_1 \in P} \sum_{p_2 \in P : p_1 < p_2} z_{{p_1}{p_2}} &\leq K,
\label{eq8} \\
    \sum_{p_2 : p_1 < p_2} z_{{p_1}{p_2}} &\leq 1, &\quad \forall p_1 \in P,
\label{eq9} \\
    \sum_{p_1 : p_1 < p_2} z_{{p_1}{p_2}} &\leq 1, &\quad \forall p_2 \in P,
\label{eq10} \\
    z_{{p_1}{p_2}} &\leq x_{p_1}, &\quad \forall p_1, p_2 \in P, \quad p_1 < p_2,
\label{eq11} \\
    z_{{p_1}{p_2}} &\leq x_{p_2}, &\quad \forall p_1, p_2 \in P, \quad p_1 < p_2,
\label{eq12} \\
    x_p &\in \{0,1\}, &\quad \forall p \in P,
\label{eq13} \\
    y^p_i &\in \{0,1\}, &\quad \forall i \in V, \quad \forall p \in P,
\label{eq14} \\
    z_{{p_1}{p_2}} &\in \{0,1\}, &\quad \forall p_1, p_2 \in P,  \quad p_1 < p_2
\label{eq15}
\end{align}
\end{spreadlines}

%Objective function (\ref{eq1}) is to minimize the size of the required fleet, while also minimizing the routing cost.  We prioritize optimizing the fleet size by assigning a weight $\theta$. Another approach for such multi-objective problems is a lexicographic, or hierarchical method (see Arora, 2012 \cite{arora2017multi}), in which preferences are imposed by ordering the objective functions.

Objective function (\ref{eq1}) minimizes the overall route cost.  It is worth noting that \cite{franccois2019adaptive} argues that the objective function of optimizing the travel time causes an unrealistic increase in waiting time.  Therefore, they discuss a different objective function of minimizing total driver working duration that includes not only vehicle travel time but also vehicle waiting time, loading time at the depot, and service times of customers.  Constraints (\ref{eq2}) make sure that if a customer $i \in V$ is served in trip $p \in P$, then trip $p$ must pass customer $i$ and belong to the optimal solution. Constraints (\ref{eq3}) guarantee exactly-once service at each customer.  Constraints (\ref{eq4}) are the capacity restrictions for each $p \in P$.  Constraints (\ref{eq5}) and (\ref{eq6}) guarantee schedule feasibility.  Constraints (\ref{eq7}) ensure no overlapping time between two successive trips of the same vehicle.  Constraint (\ref{eq8}) sets a limit on the maximum number of available vehicles.  Constraints (\ref{eq9}) and (\ref{eq10}) make sure that a trip can be immediately followed by at most one trip, and can immediately follow at most one trip, respectively.  Constraints (\ref{eq11}) and (\ref{eq12}) ensure that $z_{{p_1}{p_2}}$ can take the value of one only if both trips $p_1$ and $p_2$ are selected in the optimal solution.  Constraints (\ref{eq13})-(\ref{eq15}) restrict variable domains.
\newline

% In Constraints (\ref{eq8}), the number of vehicles used in a solution (i.e., the left side of the inequalities) is calculated as the difference between the number of trips and the number of $z_{{p_1}{p_2}}$ decision variables set to one (note that for each vehicle, the number of pairs of consecutive trips is always equal to the number of trips performed by that vehicle minus one).

%Note that the above model is a path-based formulation.  Researchers have also proposed arc-based formulation (e.g., Hernandez et al., 2014 \cite{hernandez2014new}), structure-based formulation (e.g., Paradiso et al., 2020 \cite{paradiso2020exact}), and formulations that incorporate both path and arc elements (e.g., Azi et al., 2010 \cite{azi2010exact}, Macedo et al., 2011 \cite{macedo2011solving}) to model MTVRPTWs.
% \newline

%The above formulation is an extension to Azi et al., 2010 \cite{azi2010exact}.  Azi et al., 2010 \cite{azi2010exact} denoted the set of trips used in a solution as $R$; however, it is unclear whether this set is given or how to obtain it.  In our formulation, we use $P$ and $x_p$ to denote the set of all feasible elementary trips and whether a trip is selected in the optimal solution, respectively.
%\newline


\subsection{Instances for the MTVRPTW}

To analyze the performance of the MTVRPTW algorithms, a popular choice is to adopt the widely recognized Solomon instances designed for the VRPTW \citep{solomon1987algorithms}.  The instances are \add{grouped based on the distribution of the customer locations: uniformly random (R), clustered (C), or mixed (RC)}.  Each group is additionally split into two sub-groups (1: the short horizon and short time windows; 2: the long horizon and long time windows).  Articles in the MTVRPTW literature generally exclude the first group of instances because the short scheduling horizon significantly limits the creation of journeys with multiple trips.  While each Solomon instance contains 100 customers, the studies with exact algorithms normally use only the first 25, 40, or 50 customers due to the limitation of the exact approaches.  However, studies with heuristic algorithms usually evaluate full-size instances.  \cite{gehring1999parallel} propose extensive sets of instances with up to 1000 customers that have also been used to benchmark heuristic algorithms for the MTVRPTW \citep[see][]{cattaruzza2014iterated}.  An exhaustive list of the above instances can be found online at \cite{sintef}.  Note that previous studies typically extend the above instances with problem-specific constraints and parameters to benchmark different MTVRPTW variants.

\section{Classification of MTVRPTW variants}
\label{sec:variants}

The MTVRPTW defined in Section \ref{sec:formulation} can be considered as a base case for various studies within the MTVRPTW literature because they add at least one more feature to the MTVRPTW.  This section identifies and classifies these MTVRPTW variants. The classification obtained is shown in Table \ref{table:1}.  In what follows, we introduce the motivation, formal definition, and mathematical model, modified from the model (\ref{eq1})-(\ref{eq15}) if possible, of each variant.

\begin{table}[]
\scriptsize
    \centering
    \begin{threeparttable}
    \begin{tabular}{@{}>{\raggedright}p{3cm}>{\raggedright}p{4cm}p{9cm}@{}}
    \toprule
         Paper  &   Features  &   Author's naming   \\
         \midrule
             \cite{battarra2009adaptive}
             & 1, 3, 5, 6, 7
             & Minimum Multiple Trip Vehicle Routing Problem (MMTVRP) \\
         \midrule
             \cite{azi2010exact}
             & 2, 4, 6
             & Vehicle Routing Problem with Time Windows and multiple use of vehicles \\
         \midrule
             \cite{macedo2011solving}
             & 2, 4, 6
             & Vehicle Routing Problem with Time Windows and Multiple Routes (MVRPTW) \\
         \midrule
             \cite{macedo2012generalized}
             & 2, 4, 6
             & Vehicle Routing Problem with Time Windows and Multiple Routes (MVRPTW) \\
         \midrule
             \cite{hernandez2014new}
             & 4, 6
             & Multi-Trip Vehicle Routing Problem with Time Windows and Limited Duration (MTVRPTW- LD) \\
         \midrule
             \cite{wang2014metaheuristic}
             & 2, 4, 6, 7
             & Vehicle Routing Problem with Multiple Trips and Time Windows (VRPMTW) \\
         \midrule
             \cite{azi2014adaptive}
             & 2, 4, 6
             & Vehicle Routing Problem with Multiple Routes (VRPMTW) \\
         \midrule
             \cite{cattaruzza2014iterated}
             & 1, 3, 5, 6, 7
             & Multi-Commodity Multi-Trip Vehicle Routing Problem with Time Windows (MMTVRP) \\
         \midrule
             \cite{karoonsoontawong2015efficient}
             & 7, variable customer service time
             & Multitrip Vehicle Routing Problem with Time Windows and Shift Time Limits (MTVRPTW- STL) \\
         \midrule
             \cite{hernandez2016branch}
             & 5
             & Multi-Trip Vehicle Routing Problem with Time Windows (MTVRPTW) \\
         \midrule
             \cite{anaya2016biomedical}
             & 7, 11
             & Biomedical Sample Transportation Problem (BSTP) \\
         \midrule
             \cite{cattaruzza2016multi}
             & 8
             & Multi-Trip Vehicle Routing Problem with Time Windows and Release Dates (MTVRPTW-R) \\
         \midrule
             \cite{despaux2016multi}
             & 1, 9, fleet size constraint
             & Multi-Trip Vehicle Routing Problem with Time Windows and Heterogeneous Fleet (MTVRP- TWHF) \\
         \midrule
             \addb{\cite{christiansen2017operational}}
             & \addb{10, 11}
             & \addb{Fuel Supply Vessel Routing Problem (FSVRP)} \\
         \midrule
             \cite{benkebir2019multi}
             & 7, 11
             & Multi-Trip Vehicle Routing Problem with Time Windows integrating European and French Driver Regulations \\
         \midrule
             \cite{franccois2019adaptive}
             & 4, 7, minimizing working duration
             & Multitrip Vehicle Routing Problem with Time Windows (MTVRPTW) \\
         \midrule
             \cite{paradiso2020exact}
             & 5, 6, 8
             & Capacitated Multitrip Vehicle-Routing Problems with Time Windows (CMTVRPTW) \\
         \midrule
             \cite{neira2020new}
             & 4, 6
             & Multi-Trip Vehicle Routing Problem with Time Windows, Service-Dependent loading times, and Limited Trip duration (MTVRPTW- SDLT) \\
         \midrule
             \cite{zhen2020multi}
             & 8, multi-depot
             & Multi-Depot Multi-Trip Vehicle Routing Problem with Time Windows and Release Dates (Multi-D\&T VRPTW-R) \\
         \midrule
             \cite{pan2021multi}
             & 4, 6, 10
             & Multi-Trip Time-Dependent Vehicle Routing Problem with Time Windows (MT-TDVRPTW) \\
         \midrule
             \cite{huang2021multi}
             & 11
             & Multi-Trip Vehicle Routing Problem with Time Windows and Unloading Queue at Depot (MTVRPTW-UQD) \\
         \midrule
             \cite{yang2023exact}
             & 5, 6, 8
             & Capacitated Multitrip Vehicle-Routing Problems with Time Windows (CMTVRPTW) \\
         \bottomrule
    \end{tabular}
    \begin{tablenotes}
        \item[1] Minimum MTVRPTW 
        \item[2] Profit MTVRPTW
        \item[3] Multi-Commodity MTVRPTW
        \item[4] MTVRPTW with Service-Dependent Loading Time 
        \item[5] MTVRPTW with Load-Dependent Loading Time 
        \item[6] MTVRPTW with Limited Trip Duration
        \item[7] MTVRPTW with Limited Multi-Trip Duration
        \item[8] MTVRPTW with Release Date
        \item[9] MTVRPTW with Heterogeneous Fleet
        \item[10] Time-Dependent MTVRPTW
        \item[11] MTVRPTW with Specific Application Considerations
    \end{tablenotes}
    \caption{Classification of MTVRPTW Variants}
    \label{table:1}
    \end{threeparttable}
\end{table}

\subsection{Minimum MTVRPTW}
\label{subsec:minimum}

The \textit{Minimum MTVRPTW} (M-MTVRPTW) removes the fleet size constraints and focuses on the strategic fleet sizing, with variable travel cost as the tie-breaker \cite[e.g.,][]{battarra2009adaptive, cattaruzza2014iterated}.  This extension is motivated by the fact that vehicle cost is much higher than the variable travel cost.  We can remove constraints (\ref{eq8}) and replace objective function (\ref{eq1}) by objective function (\ref{eq16}) to formulate the M-MTVRPTW:
% For example, a delivery service operating self-driving electric delivery vans for its last-mile delivery network, or a business outsourcing its delivery to a subcontractor with a per-vehicle cost scheme, might find minimizing the number of used vehicles as the most suitable objective.

\begin{equation}
Min \quad
    \sum_{p \in P} c_p x_p - \theta \sum_{p_1 \in P} \sum_{p_2 \in P : p_1 < p_2} z_{{p_1}{p_2}} \label{eq16}
\end{equation}
\begin{align}
s.t. \quad (\ref{eq2})-(\ref{eq7})&, (\ref{eq9})-(\ref{eq15}) \nonumber
\end{align}

The objective function (\ref{eq16}) optimizes both the fleet size and the route cost.  A parameter $\theta$ is introduced to determine the relative weight between the two objectives.  Note that aside from the weighted sum method, the lexicographic and hierarchical methods are also commonly used to tackle multi-objective optimization problems \citep[e.g.,][]{arora2017multi}, in which preferences are established by arranging the objective functions in a specific order.

\subsection{Profit MTVRPTW}

A limited number of vehicles with capacity constraints may not be able to service all customers within their specified service time windows.  Hence, the \textit{Profit MTVRPTW} (P-MTVRPTW) relaxes the constraint that all customers must be served. It follows that the objective of the P-MTVRPTW, which is similar to the Orienteering Problem, is to maximize the profit defined as the revenue obtained by serving selected customers minus the total travel cost.  Let $g_i$ be the revenue associated with customer $i \in V$, the P-MTVRPTW can be formulated as follows.
\begin{equation} \ \label{eq17}
    Max \quad \sum_{i \in V} \sum_{p \in P} g_i y^p_i - \sum_{p \in P} c_p x_p
\end{equation}
\begin{equation} \label{eq18}
    s.t. \quad \sum_{p \in P} y^p_i \leq 1, \quad \forall i \in V
\end{equation}
\begin{align}
(\ref{eq2})&, (\ref{eq4})-(\ref{eq15}) \nonumber
\end{align}

As indicated, the objective function (\ref{eq17}) maximizes the profit. Constraints (\ref{eq18}) replace constraints (\ref{eq3}), and ensure that a customer can be unserved. It is noted that most previous studies consider the case that all customers are associated with the same revenue by setting $g_i = 1, \forall i \in V$ \citep[e.g.,][]{azi2010exact, macedo2011solving, macedo2012generalized, wang2014metaheuristic}.


\subsection{Multi-commodity MTVRPTW}
The \textit{Multi-Commodity MTVRPTW} (MC-MTVRPTW), first introduced by \cite{battarra2009adaptive}, assigns a commodity set $C$ to a customer set $V$ while imposing that a vehicle cannot carry goods of different commodities together.  The MC-MTVRPTW can be formulated by introducing commodity-specific parameters:

\begin{itemize}
    \item Each customer $i \in V$ is associated with $C$ different demands, with each demand having an independent service time and time window.  More specifically, customer $i$'s demand for a commodity $c$ is denoted as $q_{ic}$ that is correlated with a service time $s_{ic}$, and a time window $[a_{ic},b_{ic}]$.
    \item Each vehicle has a capacity $Q_c$ and a cost factor $\tau_c$ per travel unit for commodity $c \in C$.
    \newline
    \textit{(We omit other constraints for brevity)}
\end{itemize}

\cite{battarra2009adaptive} formulate the supermarket goods distribution problem on a territory spanning multiple regions as an MC-MTVRPTW that involves three different commodities of vegetables (V), fresh products (F), and non-perishable items (N).  Additionally, the routing problem considered by \cite{battarra2009adaptive} involves not only multi-commodity but also fleet size minimization, and is referred to as the Multi-Commodity Minimum MTVRPTW (MC-M-MTVRPTW) in this review.

\subsection{MTVRPTW with dependent loading or unloading times}
% In the \textit{MTVRPTW with Service-Dependent Loading Times} (MTVRPTW-SDLT), the vehicle loading time at the depot of a trip is dependent on the total service times of customers.  Azi et al. (2010) \cite{azi2010exact} introduced the following constraints to model the MTVRPTW-SDLT:

%(i.e., service-dependent) or the total customer load in that trip (i.e., load-dependent).  Azi et al. (2010) \cite{azi2010exact} introduced the following constraints to model the \textit{MTVRPTW with Service-Dependent Loading Times} (MTVRPTW-SDLT)\footnote{Not to be confused with the abbreviation suggested in Neira et al. (2020) \cite{neira2020new} for the MTVRP with Time Windows, Service Dependent loading times, and Limited Trip duration}:

Two types of the \textit{MTVRPTW with Dependent Loading Times} can be found in the literature.  One type is the \textit{MTVRPTW with Service-Dependent Loading Times} (MTVRPTW-SDLT), in which the vehicle loading time of a trip depends on the sum of customers' service time in the trip.  In such a setting, a vehicle cannot begin the trip before the corresponding loading at the depot is completed. That is, by adding constraints (\ref{eq19}) to and using constraints (\ref{eq20}) and (\ref{eq21}) to replace constraints (\ref{eq7}) in model (\ref{eq1})-(\ref{eq15}), the MTVRPTW-SDLT can be formulated as follows \citep{azi2010exact}:

\begin{spreadlines}{14pt} % ooiwt - from mathtools
\allowdisplaybreaks
\begin{align} 
Min &\quad \sum_{p \in P} c_p x_p &\quad
\tag{\ref{eq1}}
\end{align}
\begin{align}
    s.t. \quad \sigma^p &= s_0 + \tau \sum_{i \in V} s_i y^p_i, & \quad \forall p \in P,
\label{eq19} \\
    t^p_0 &\geq a_0 + \sigma^p, & \quad \forall p \in P,
\label{eq20} \\
    t^{p_1}_{n + 1} + \sigma^{p_2} &\leq t^{p_2}_0 + M(1 - z_{{p_1}{p_2}}), & \quad \forall p_1, p_2 (p_1 < p_2) \in P,
\label{eq21} \\
(\ref{eq2})-(\ref{eq6})&, (\ref{eq8})-(\ref{eq15}) \nonumber
\end{align}
\end{spreadlines}

where $\sigma^p$ indicates the total loading time for trip $p \in P$; $s_0$ is the constant loading time (i.e., setup time) at the depot; $\tau \geq 0$ is the loading factor representing the variable loading time needed per service time unit in the trip.  Note that similar to the MTVRPTW, most previous MTVRPTW-SDLT studies exclude loading time from trip duration \citep[e.g.,][]{macedo2011solving, hernandez2014new}.
\newline

%In such settings, each trip cannot start before the vehicle is fully loaded at the depot. This restriction is illustrated in Constraints (\ref{eq20}) and (\ref{eq21}), which replace Constraints (\ref{eq7}) in the MTVRPTW formulation.
%\newline

%The \textit{MTVRPTW with Load-Dependent Loading Times} (MTVRPTW-LDLT) is a variant of the MTVRPTW-SDLT, which accounts for the quantity of cargo delivered in the trip to the loading time.  Battarra et al. (2009) \cite{battarra2009adaptive} attributed the loading time required at the depot to a constant maneuver time and a unit loading time for each unit of goods to be delivered in the trip.  The fixed and variable components of the loading time can be illustrated with the following constraints (replacing Constraints (\ref{eq19}) in the MTVRPTW-SDLT formulation):

The other type is the \textit{MTVRPTW with Quantity-Dependent Loading Times} (MTVRPTW-QDLT) which accounts for the quantity of cargo delivered in the trip to the loading time at the depot.  \cite{battarra2009adaptive} define this loading time to be the sum of a constant operation time plus the time it takes to load all cargo delivered in the trip.  We can modify the mathematical model of the MTVRPTW-SDLT by replacing constraints (\ref{eq19}) with constraints (\ref{eq22}) to have the model of the MTVRPTW-QDLT.

\begin{equation} \label{eq22}
    \sigma^p = s_0 + \tau \sum_{i \in V} q_i y^p_i, \quad \forall p \in P
\end{equation}

Likewise, the unloading time may also be treated as a variable. That is, the \textit{MTVRPTW with Quantity-Dependent Unloading Times} (MTVRPTW-QDUT) assumes that the unloading times at the depot and/or customers are variable instead of fixed.  \cite{karoonsoontawong2015efficient} assumes that the unloading time at each customer is calculated by multiplying the customer demand by the delivery rate.  The author does not develop mathematical models but directly resorts to heuristic algorithms.  Inspired by challenges in urban waste collection, \cite{huang2021multi} associates the depot with a restricted unloading capacity.  Hence, if the unloading capacity is fully used, certain vehicles must be put in a queue, and the vehicle unloading time at the depot is determined by the product of the unit unloading time of cargo and the number of carried cargo.  To mathematically model their MTVRPTW-QDUT, they consider a set of discrete unloading time slots, with each associated with the maximum number of vehicles at the depot.

\subsection{MTVRPTW with limited trip duration}

% The \textit{MTVRPTW with Limited Trip Duration} (MTVRPTW-LTD) imposes a trip duration limit, where the service of the last customer in the trip cannot start later than $t_{max}$ time units after the trip begins. These deadline constraints are relevant in the case of delivering perishable goods, which must be delivered within a certain amount of time from loading. Additionally, this trip-duration limit, which leads to the creation of short trips, is also considered as the motivation for making multiple trips in some MTVRPTW papers (e.g., Azi et al., 2010 \cite{azi2010exact}; Azi et al., 2014 \cite{azi2014adaptive}).  The MTVRPTW-LTD can be modeled by adding the following constraints to the basic MTVRPTW formulation.

Two types of the \textit{MTVRPTW with Limited Trip Duration} (MTVRPTW-LTD) exist in the routing literature.  One type is the \textit{MTVRPTW with Limited Single-Trip Duration} (MTVRPTW-LSTD) that imposes a trip duration limit, where the service of the last customer in the trip cannot begin later than $t_{max}$ time units after the trip starts.  These maximum trip duration constraints are commonly considered in routing problems with perishable products.  Note that the trip-duration limit leads to the creation of short trips, and is thus being considered as the motivation for allowing vehicles to make multiple trips by some articles \citep[e.g.,][]{azi2010exact, azi2014adaptive}.  The MTVRPTW-LSTD can be formulated by incorporating additional constraints to the basic MTVRPTW formulation listed as program (\ref{eq1})-(\ref{eq15}).  The additional constraints are:

\begin{equation} \label{eq23}
    t^p_i \leq t^p_0 + t_{max}, \quad \forall i \in V, \quad \forall p \in P
\end{equation}

where the trip duration limit $t_{max}$ excludes the time for loading the vehicle, servicing the last customer, and returning to the depot.
\newline

% A variant of the MTVRPTW-LTD, the \textit{MTVRPTW with Limited Multi-Trips Duration} (MTVRPTW-LMTD) restricts the multiple-trip (i.e., journey) duration from departure at the depot of the first trip to arrival at the depot of the last trip to be no larger than $D_{max}$.  This multiple-trip duration, sometimes referred to as "spread time" (e.g., Battarra et al., 2009 \cite{battarra2009adaptive}, Wang et al., 2014 \cite{wang2014metaheuristic}, Karoonsoontawong, 2015 \cite{karoonsoontawong2015efficient}, Cattaruzza et al., 2016b \cite{cattaruzza2016vehicle}) or "driver shift" (e.g., François et al., 2019 \cite{franccois2019adaptive}), represents the driver's working shift length, which is often limited by regulations.  We can capture the LMTD characteristic with the following additional constraints:

The other type is the \textit{MTVRPTW with Limited Multi-Trip Duration} (MTVRPTW-LMTD) that restricts the multiple-trip (i.e., the whole journey) duration of each vehicle.  That is, the time that passes from the departure of the first trip from the depot until the arrival of the last trip at the depot cannot be larger than $D_{max}$.  This multiple-trip duration is also referred to as "spread time" \citep[e.g.,][]{battarra2009adaptive, wang2014metaheuristic, karoonsoontawong2015efficient, cattaruzza2016vehicle} or "driver shift" \citep[e.g.,][]{franccois2019adaptive} that represents a driver's working shift length, often limited by regulations.  Assuming that all used vehicles start their first trips from the depot at the same time, e.g., $a_0$, the mathematical formulation of the MTVRPTW-LMTD can be derived by introducing the following constraints into the program (\ref{eq1})-(\ref{eq15}).

\begin{equation} \label{eq24}
    t^p_{n+1} - a_0 \leq D_{max}, \quad \forall p \in P
\end{equation}

\subsection{MTVRPTW with release date}

The \textit{MTVRPTW with Release date} (MTVRPTW-R) associates each customer with a release date, denoting when the goods allocated for the customer are ready for transport at the depot.  Let $r_i$ be the release date associated with the demand of customer $i$.  The release date constraints can be described as:

\begin{equation} \label{eq25}
    r_i \leq t^p_0 + M(1 - y^p_i), \quad \forall p \in P
\end{equation}

\cite{cattaruzza2016multi} model the last-mile delivery problem involving city distribution centers (CDC) as the MTVRPTW-R, in which packages are first transported to the CDC before delivering to customers.  \cite{zhen2020multi} consider not only the release date but also multi-depot for the practical operations of online shopping package delivery.  They assume that each depot is associated with its own vehicle fleet, and each trip must begin and end at the same depot.

\subsection{MTVRPTW with heterogeneous fleet}

The \textit{MTVRPTW with Heterogeneous Fleet} (MTVRPTW-HF) considers a mixed fleet of vehicles, each having distinct capacities and travel costs.  \cite{despaux2016multi} addresses a MTVRPTW-HF version that associates a distinct fixed cost to each vehicle, together with the fleet size constraints, which differs their problem from the M-MTVRPTW.  To mathematically model the MTVRPTW-HF, standard parameters such as vehicle capacity, travel times, and travel costs must be replaced by individual parameters.  Variables must be adjusted accordingly.

\subsection{Time-dependent MTVRPTW}

The \textit{Time-Dependent MTVRPTW} (TD-MTVRPTW) relaxes the conventional assumption that travel time between two nodes is constant.  \cite{pan2021multi} investigate an urban routing problem, in which the speed of vehicles varies based on the time of departure.  The authors capture the time-dependent aspect by dividing the workday into time slots and then modeling the speed profile as a stepwise function.

\subsection{MTVRPTW with specific application considerations}

The \textit{MTVRPTW with Specific Application Considerations} (MTVRPTW-SAC) incorporates various specific considerations into the MTVRPTW.  \cite{anaya2016biomedical} research the biomedical sample transportation problem, in which they remove vehicle number and capacity constraints while introducing multiple pick-ups at the same node, maximal transportation time after a sample is collected, and exceptions for emergency requests.  \addb{\cite{christiansen2017operational} study a maritime transportation planning problem with stowage constraints and dynamic sailing times, in which a heterogeneous fleet of supply vessels makes multiple trips to service different fuel types to time-window constrained customer ships that are anchored outside a major port.}  \cite{benkebir2019multi} consider the MTVRPTW with multiple driver regulations, including compulsory breaks and rest periods, daily and weekly working time limitation, and exceptions for extension of longer working time or shorter rest periods.


%\section{MTVRPTW Algorithms}
%\label{sec:algorithms}

%In this section, we will review both exact and heuristic approaches for the MTVRPTW and its variants.  We will also discuss adaptations of the solution approaches for specific variants, where applicable.
%\newline

\section{Exact algorithms}
\label{sec:exact}

The VRP and MTVRP have been established as NP-hard problems \citep[see, e.g.,][]{lenstra1981complexity, olivera2007adaptive}. 
 It follows that the MTVRPTW is also NP-hard, as the MTVRP is reducible to the MTVRPTW by setting the time window at each customer to $[0, \infty]$.  In Section \ref{sec:formulation}, we model the MTVRPTW as a MILP.  Hence, one can utilize commercial optimization solvers like IBM ILOG CPLEX and Gurobi to tackle small-scale MTVRPTWs.  However, for most medium-scale and large-scale instances, more efficient solution algorithms are called for.  This section reviews the exact algorithms suggested for the MTVRPTW and its variants.  The heuristic algorithms are surveyed in the next section.
\newline

\add{\textit{Branch-and-cut}, \textit{branch-and-price}, and their combinations are popular exact algorithms for solving the MILPs. Branch-and-cut tries to obtain the optimal solution of a MILP through the application of a \textit{branch-and-bound}  algorithm while incorporating cutting planes to tighten the linear programming relaxations \citep{padberg1991branch}.  The effectiveness of the cutting plane procedure depends on finding a violated valid inequality in the relaxed MILP, which is often termed as the \textit{separation problem}.  Furthermore, \cite{barnhart1998branch} propose the branch-and-price, which integrates \textit{column generation} with branch-and-bound.  More specifically, the column generation method is characterized by a \textit{master problem} (MP) that is the initial problem with only a subset of the variables taken into account, and a \textit{pricing subproblem} that determines an improving variable (i.e., improving the objective function of the MP).  The branch-and-bound in the above optimization methods divides an optimization problem into smaller sub-problems and uses a bounding function to prune sub-problems that cannot have the optimal solution.}
\newline

%\footnote{We omit Neira et al. (2020) \cite{neira2020new} because its two MILP formulations do not fit with the table structure.}

% A classic approach for solving MILPs is \textit{Branch-and-cut} (Padberg and Rinaldi, 1991 \cite{padberg1991branch}), which tries to find an optimal solution through relaxing the integrality restrictions while using \textit{cutting planes} to obtain a lower bound for the original problem (i.e., tighten the LP relaxation).  The effectiveness of the cutting plane procedure depends on finding a violating inequality in the relaxed MILP's solution, which is commonly referred to as \textit{separation problem}.  Another method, which is effective in dealing with MILPs having a huge number of variables, is \textit{Column generation} (Ford and Fulkerson, 1958 \cite{ford1958suggested}).  The column generation method is characterized by a \textit{master problem} (MP), which is the original problem with only a subset of variables being considered, and a \textit{pricing subproblem}, which is a new problem created to determine an improving variable (i.e., improve the objective function of the MP).  Barnhart et al. (1998) \cite{barnhart1998branch} proposed the \textit{Branch-and-price} method that combines column generation and \textit{branch-and-bound}.  The branch-and-bound method is similar to branch-and-cut, which also tries to relax the integrality restrictions, except that it does not involve cutting planes.
%\newline

\cite{azi2010exact} employ the branch-and-price method to tackle the Profit MTVRPTW with Service-Dependent Loading Times and Limited Trip Duration (P-MTVRPTW-SDLT-LTD).  The MP in column generation is formulated as a set packing problem whose variables (i.e., columns) represent vehicle journeys.  The pricing subproblem is modeled as an elementary shortest path problem with resource constraints (ESPPRC), which is defined on a graph with nodes representing trips and arcs representing feasible consecutive trips.  The algorithm's performance is evaluated using instances adopted from \cite{solomon1987algorithms}, specifically R2, C2, and RC2 instances (i.e., long scheduling horizon) with 25 and 40 customers.  Instances with short scheduling horizons are discarded because they prevent the vehicles from performing multiple trips.  \cite{hernandez2014new} present an algorithm based on branch-and-price for the MTVRPTW with Service-Dependent Loading Times and Limited Trip Duration (MTVRPTW-SDLT-LTD).  They formulate the MP as a set covering problem, whose variables (i.e., columns) represent trips.  They then solve the pricing subproblem to identify new trips with negative reduced costs.  In addition, they use mutual exclusion constraints to assign a selected trip to vehicles, while preventing assigning two overlapping trips to the same vehicle. 
 \cite{hernandez2016branch} propose two branch-and-price algorithms, each founded on a different set covering formulation, to solve the MTVRPTW-LDLT.  One of the formulations considers trips as columns and the other treats journeys as columns.  After comparing the two algorithms using Solomon's benchmark \citep{solomon1987algorithms}, they conclude that the trip-based representation is generally more effective.  \cite{neira2020new} present two distinct MILP formulations for both the MTVRPTW-SDLT-LTD and the MTVRPTW-SDLT, and show that model formulation has a significant effect on model performance.  More specifically, they show that their models outperform the three-index formulations for the MTVRPTW-SDLT and MTVRPTW-SD in the literature, and can be implemented with ease using standard optimization solvers available in the market.  In addition, in terms of the MTVRPTW-SD, their models are competitive compared to the branch-and-price algorithms presented by \cite{hernandez2016branch}.  \cite{huang2021multi} formulate their MTVRPTW-QDUT as a trip-based set partitioning model.  To address this model, they employ a branch-and-price-and-cut algorithm, which utilizes the column generation approach to handle the linear relaxation and incorporates rounded capacity inequalities to tighten the relaxation gap.
\newline

Apart from the exact algorithms mentioned above, other methods can also be found in the literature.  \cite{macedo2011solving} introduce a pseudo-polynomial network flow model designed for the P-MTVRPTW-SDLT-LTD.  In the model, each vehicle journey corresponds to a path in an acyclic-directed graph, where nodes symbolize discrete time instants and arcs symbolize feasible trips.  In addition, each time instant corresponds to a specific continuous time interval, also referred to as granularity.  The number of constraints is polynomial in the cardinality of this time-indexed graph, thus the pseudo-polynomial naming.  During execution, the algorithm iteratively refines the granularity until a feasible solution is achieved.  The same authors generalize this algorithm with new discretization rules in \cite{macedo2012generalized}.  Both articles consider the same instances outlined in \cite{azi2010exact}.  Computational results suggest that this algorithm is effective in reducing execution time for most of the test instances.
\newline

Indeed, most exact methods are tailored to specific MTVRPTW variants, and thus it is not trivial to extend those methods to solve other variants.  \cite{paradiso2020exact} introduce an exact solution framework (ESF) capable of solving four different MTVRPTW variants: the MTVRPTW-LDLT, the MTVRPTW-LTD, the MTVRPTW-R, and the Drone-Routing Problem \citep[e.g.,][]{cheng2018formulations}.  The ESF relies on the notion of \textit{structure}, which is a trip with a starting time interval such that the trip duration and cost are the same for any departure times in this interval.  This structure-based formulation greatly decreases the count of variables and constraints compared to other formulations.  The ESF employs a branch-and-cut procedure, whose separation problem is to determine if a structure set represents a feasible MTVRPTW solution.  This separation problem is modeled as a Team Orienteering Problem with Time Windows \citep[see, e.g.,][]{vansteenwegen2009iterated} and then solved by the column generation method.  Experimental results indicate that the ESF surpasses previous works in the literature.  \cite{yang2023exact} enhances the ESF \citep{paradiso2020exact} to address instances involving as many as 70 customers.
\newline

It is important to note that the majority of exact algorithms for the MTVRPTW require an enumeration of all feasible non-dominated trips.  \cite{azi2007exact} introduce a dominance concept for the single-vehicle routing problem, with time windows and multiple routes to prematurely discard non-promising partial \textit{path}.  A path $p_1$ is considered to dominate another path $p_2$ under the following conditions: (1) They both end at the same customer; (2) they contain the same customer set, possibly in a different sequence; and (3) $p_1$ is not longer and does not require more resources than $p_2$.  In \cite{azi2007exact}, a path can be a trip, journey, or structure, depending on the context.  This dynamic programming-based approach is adopted by subsequent articles to accelerate the search.  For example, \cite{macedo2011solving} propose three additional dominance rules for their network flow model to reduce the arc count \addb{for the P-MTVRPTW-SDLT-LTD}, while \cite{hernandez2016branch} and \cite{christiansen2017operational} develop \addb{new} dominance relations for \addb{the MTVRPTW-LDLT and the TD-MTVRPTW-SAC, respectively}.  As noted by \cite{azi2007exact}, this dominance method is particularly sensitive to deadline constraints (i.e., the number of feasible routes increases drastically when the trip duration limit is not tight).  That is, this dominance method may struggle with less time-constrained MTVRPTW variants.
\newline

A summary of exact algorithms for the MTVRPTW and its variants is presented in Table \ref{table:2}.

\begin{table}[]
\small
    \centering
    % \begin{tabular}{@{}>{\raggedright}p{1.8cm}>{\raggedright}p{3cm}>{\raggedright}p{4.5cm}l@{}}
    \begin{tabular}{@{}>{\raggedright}p{3.5cm}>{\raggedright}p{4cm}p{6cm}@{}}
    \toprule
         Paper  &   Method  &   Model   \\
         \midrule
         \cite{azi2010exact}
         & Branch-and-price
         & Master problem: set packing
            \newline Pricing subproblem: elementary shortest-path problem with resource constraints (ESPPRC) \\
         \midrule
         \cite{macedo2011solving}
         & Network flow, time discretization
         & Minimum flow problem
            \newline Solution: paths in network flow
            \newline Nodes: discrete time instants \\
         \midrule
         \cite{macedo2012generalized}
         & Network flow, time discretization
         & Minimum flow problem
            \newline Solution: paths in network flow
            \newline Nodes: discrete time instants \\
         \midrule
         \cite{hernandez2014new}
         & Branch-and-price
         & Master problem: set covering
            \newline Pricing subproblem: select timing for trips \\
         \midrule
         \cite{hernandez2016branch}
         & Branch-and-price
         & Master problem: set covering
            \newline Pricing subproblem: ESPPRC \\
         \midrule
         \addb{\cite{christiansen2017operational}}
         & \addb{Branch-and-bound}
         & \addb{Arc-flow and path-flow models} \\
         \midrule
         \cite{paradiso2020exact}
         & Branch-and-cut with embeded column generation
         & (Separation problem of branch-and-cut) Team Orienteering Problem with Time Windows \\
         \midrule
         \cite{neira2020new}
         & MILP formulations (solve with CPLEX)
         & Two-index node insertion model,
            \newline Two-index arc insertion model \\
         \midrule
         \cite{huang2021multi}
         & Branch-and-price
         & Master problem: set partitioning
            \newline Pricing subproblem: ESPPRC \\
         \midrule
         \cite{yang2023exact}
         & Price-cut-and-enumerate
         & (Separation problem of branch-and-cut) Team Orienteering Problem with Time Windows \\
         \bottomrule
    \end{tabular}
    \caption{Exact Algorithms}
    \label{table:2}
\end{table}


%Note that, in Hernandez et al. (2016) \cite{hernandez2016branch}, both formulations (i.e., journey and trip representations) have the same model.  Additionally, the last column of the table denotes the variable in the column representation of the algorithms.  Note that this is different from the variable in the mathematical model of the problem (i.e., arc-based or path-based formulation).

\section{Heuristic algorithms}
\label{sec:heuristics}

Although recent developments in exact solution algorithms for the MTVRPTW can handle instances involving up to 70 customers \citep{yang2023exact}, the complexity of exact algorithms is inherently exponential of the input size.  Large-sized problem instances of the MTVRPTW quickly become intractable for exact algorithms, \add{thus, require heuristic algorithms}.  Similar to the exact algorithms, heuristic algorithms for the MTVRPTW are often designed for different variants.  These heuristics, however, share a general framework: start with a solution construction phase (or routing phase), followed by an improvement phase, often a metaheuristic, which guides \textit{local search} (LS) moves to achieve global optimum.  \add{We note that the methods employed in the routing phase of the MTVRPTW are similar to those in the VRP and VRPTW, which have been extensively studied. Therefore, we omit the routing phase's methods and only provide a summary in Table \ref{table:3} for reference.}  In this section, we first review some articles that adapt heuristics of related problems in the literature to tackle the MTVRPTW.  We then analyze challenges in adapting existing heuristics in the context of the MTVRPTW, namely the time window and multi-trip characteristics.  We omit special techniques to handle specific variants and leave that as an open research problem.  Lastly, we discuss an aspect of the search heuristic that allows the solution to become infeasible.
\newline

%  \cite{yuan2021column} propose a column generation based heuristic for the Generalized VRPTW, which solves a restricted master problem on a subset of all feasible routes.

Heuristics for MTVRPTWs are usually \addb{derivative, in other words, built upon existing ideas.}  \cite{karoonsoontawong2015efficient} modifies the efficient insertion heuristic originally designed for the \add{VRPTW with limited trip duration} by \cite{campbell2004efficient} to generate solutions with multiple trips per vehicle for the MTVRPTW-LMTD with variable customer service times.  \cite{battarra2009adaptive} adopt a routing-packing decomposition method commonly seen in the VRPTW literature \citep[e.g.,][]{fleischmann1990vehicle, taillard1996vehicle} to solve another MTVRPTW variant.  In this paper, the routing phase generates a set of feasible trips using a well-known procedure for the VRP \citep{christofides1976vehicle}, while the packing phase combines trips into journeys by a simple greedy algorithm.  \cite{cattaruzza2014iterated} introduce an iterated local search (ILS) algorithm, in which the solution is a \textit{giant tour} (i.e., permutation of customers) and the perturbation is the crossover operator from the \textit{genetic algorithm} (GA) metaheuristic.  The GA has been effectively utilized to address the VRP and the MTVRP \citep{prins2004simple, cattaruzza2014memetic}.  To obtain an MTVRPTW solution from the giant tour, \cite{cattaruzza2014iterated} perform an AdSplit procedure, which is inspired by a similar procedure for the VRP \citep{prins2004simple}.
\newline

% Talk about solution construction heuristic? parallel journey construction, split operator (cite: Azi et al. (2014 \cite{azi2014adaptive}))?

Nevertheless, there are challenges in adapting the heuristics of related problems to tackle the MTVRPTWs.  \add{We classify the challenges into those caused by (i) adding time window constraints and (ii) allowing multiple trips and discuss the approaches used by existing work to tackle these challenges.}

\subsection{Time windows}
First, the presence of time windows complicates travel cost calculation and feasibility check because the impact of any trip modifications may propagate to the rest of the trip and vehicle journey.  To overcome such cascading effect, \cite{cattaruzza2016multi} propose a scheme to evaluate classical LS moves for the MTVRPTW-R in constant time by maintaining a list of special quantities \add{from \cite{vidal2013hybrid}}.  Assessing the feasibility of trips is further complicated by the time-dependent context (i.e., variable loading time, service time, travel time, etc.).  Indeed, \cite{cattaruzza2016multi} note that the evaluation of time window violation is linear to the \add{maximum} count of trips \add{within all journeys} when the vehicle loading time is trip-dependent.  \add{The time-dependent variant of MTVRPTW (TD-MTVRPTW) adds further complexity to feasibility check as segment update is non-trivial.  To address this, \cite{pan2021multi} formulate a ready time function and duration function that are time-dependent and show how these two functions can be used to perform feasibility checks in polynomial time.}
\newline
%scheme to deal with the TD-MTVRPTW.  This extension significantly improves the time complexity of this frequently-invoked procedure.  


\add{Second}, time windows \add{may disallow packing of two trips into a journey due to overlapping time.  Such incompatible trips with overlapping time windows would require service from additional vehicles, which may lead to infeasible solutions if all the vehicles have been used.}  %\textit{incompatible} (i.e., cannot be combined due to time overlaps) \add{and thus requires additional consideration compared to the MTVRP}.   
\cite{battarra2009adaptive} introduce an adaptive guidance mechanism that discourages the creation of incompatible trips.  At each iteration, the guidance mechanism tries to \add{penalize creation of routes that overlap with the} \textit{critical} time intervals, which are intervals when many trips are strongly active (i.e., time period that the trip is active in every possible scheduling).  % and critical commodities, which are commodities that are not well-packed across multiple trips.  Then the two critical features are eliminated in subsequent iterations by means of penalization or input parameters tuning.  Although the critical commodity is distinctive to the variants with multiple commodities, the critical time interval is generally a common feature of the MTVRPTWs.  
Similarly, \cite{wang2014metaheuristic} \add{also exploits critical time intervals in their heuristic for the P-MTVRPTW.  They propose a trip partition operator %(referred to as "route partition" in the article)
that splits a trip if it overlaps with the critical time interval to create non-overlapping sub-trips.}  %any pairs of customers if  to exploit this characteristic.  
\add{Following from this, the routing-packing decomposition approach, which is effective for MTVRPs,}
%\cite{cattaruzza2016vehicle} suggest that the routing-packing decomposition, which is effective for the MTVRPs, 
struggles with the MTVRPTWs because the time window constraints make the two phases (i.e., routing and packing) deeply inter-connected \citep{cattaruzza2016vehicle}.  \add{Heuristics for MTVRPTW thus require an integrated approach that considers both routing and packing, to be more effective %as shown When comparing the effectiveness of an integrated approach with that of a routing-packing decomposition, 
\citep[e.g.,][]{franccois2019adaptive}}.
%find that an integrated solution method outperforms the two-phase approach in the presence of time windows.
\newline

% For instance, the introduction of time windows in the MTVRPTW causes modifications in an earlier trip to impact the rest of a vehicle journey.  Therefore, the approach of decomposing solution in two successive phases: a routing phase and a packing phase, which is effective for MTVRP, often lead to infeasible solutions for the MTVRPTW.

%TODO: change to present tense
\subsection{Multiple trips}
The multi-trip attribute of the MTVRPTW means that each \add{journey} may contain multiple trips, while each trip may \add{visit} multiple customers.  Consequently, the solution \add{has} a multi-layered structure: a vehicle journey layer and a trip layer \citep{wang2014metaheuristic}.  Classical customer-based operators for the VRP solution approaches like customer relocation, exchange, and insertion are insufficient to achieve good solutions for the MTVRPTW because they solely focus on the trip layer.  Therefore, to effectively tackle the multi-layered characteristic of the MTVRPTW, multi-level approaches are often necessary.  
\newline

\cite{wang2014metaheuristic} \add{adopt a pool-based metaheuristic in which they maintain a pool of feasible solutions for the vehicle journey layer.  A solution is iteratively updated through an LS procedure and a new solution is added by combining trips from solutions in the pool}. %The authors then incorporated the LS procedure into a pool-based metaheuristic in which a pool of trips is initialized and updated iteratively (i.e., focuses on the trip layer), then merged to form vehicle working schedules (i.e., focuses on the vehicle layer).  
%Although presentations differ, we believe that the algorithms in \cite{wang2014metaheuristic} and \cite{battarra2009adaptive} share the same fundamental idea: a two-phase routing-packing approach. %with a special procedure to update the trips produced in the first phase.  
\cite{cattaruzza2016multi} consider the multi-trip aspect by extending the customer relocation and exchange operators to \add{relocation of a trip (to another journey) and swapping of trips between two journeys.}  These new trip-based operators, together with classical customer-based operators, can be either inter-vehicle or intra-vehicle in the multi-trip context.  
\add{\cite{pan2021multi} also leverages trip-level operators in the LS procedure, including removing a trip, removing two consecutive trips from one journey, and two trips from different journeys.}  %The algorithm then reconstructs the solution through a look-ahead approach named regret insert.
\cite{azi2014adaptive} design an adaptive large neighborhood search (ALNS)\add{-based solution} that exploits the hierarchical structure of the MTVRPTW.  The ALNS optimizes a solution through the utilization of destruction and reconstruction operators. %\citep[also referred to as removal and repair operators in][]{pan2021multi}. 
\add{In considering the multi-layer structure of the problem, they proposed to probabilistically remove an entire vehicle's journey or a trip from the solution, in addition to the usual customer removal.}
%then submitting the result to an acceptance criterion used in SA.  The destruction operators are selected iteratively from high-level (i.e., vehicle, trip removal) operators to low-level (i.e., customer removal) operators.  At each iteration, the respective reconstruction operators, which are devised from insertion heuristics like least-cost or regret-based, are weighted based on historical performance and chosen probabilistically.  Empirical results show that this multi-level approach outperforms the classical customer-based approach.  
\newline

\subsection{Infeasible solutions}
The numerous constraints of MTVRPTWs greatly increase the likelihood of reaching infeasible solutions during the search.  \add{Current heuristics either allow infeasible intermediate solutions and attempt to repair the violations after the final solution is found, or guide the search space towards the feasible solutions when an infeasible intermediate solution is encountered.} \newline

\cite{cattaruzza2014iterated} and \cite{franccois2019adaptive} allow infeasible intermediate solutions by relaxing the original problems on the time-related constraints.  In case the final solution is invalid, \cite{cattaruzza2014iterated} executes a repair procedure to detect and fix infeasible trips, while \cite{franccois2019adaptive} %perform a variable neighborhood descent with a customized cost function that focuses on reducing any time-related infeasibilities.  
\add{perform relocation or exchange on a chain of vertices during a post-optimization phase to reduce time-related infeasbilities.}  \cite{despaux2016multi} solve the MTVRPTW-HF with a simulated annealing (SA) algorithm, whose LS moves can violate constraints related to both time window and vehicle capacity.  The algorithm then attempts to lead the search into a feasible space of solutions by prioritizing two special operations that eliminate constraint violations.  \add{If the capacity constraint is violated, the algorithm relocates the customer with the greatest demand from the violating trip to another trip with available vehicle capacity.  For violation of the time-window constraint, the algorithm relocates the last visited customer from the violating trip to another trip that ends before the specified time constraint.}    \cite{cattaruzza2016multi}, considering the MTVRPTW-R, incorporate two penalty factors into the \add{fitness} function \add{of their memetic algorithm to penalize time window and vehicle capacity violations.}  The \add{penalty factor} is adjusted every time the population of its memetic algorithm reaches a certain dimension 
\add{according to the number of violations observed in recently generated individuals.}  Although allowing infeasible intermediate solutions may expand the search space, we believe that it increases search diversity at the expense of additional runtime and algorithm complexity; thus, it should be introduced with care.

\subsection{Summary}
In practice, studies in the MTVRPTW literature usually employ a mixture of the aforementioned methods.  A comprehensive categorization can be found in Table \ref{table:3}.  In the table, we also record the heuristics of the routing phase %whether the routing phase of an algorithm produces a feasible solution (Feasible Sol. Construction column) 
and whether intermediate solutions can be infeasible (Infeasible Intmd. Sol. column), respectively.  Note that evolutionary algorithms like genetic algorithms and memetic algorithms generally start with an initial population, which can be transformed into solutions with a special procedure.  Therefore, we regard their routing phases as being able to construct a solution \citep[e.g.,][]{cattaruzza2016multi, zhen2020multi}.  Additionally, a summary of each algorithm is provided for quick reference.  \add{Some papers with specific application constraints are omitted because their problem formulations include special considerations specific to the application context \citep[e.g.,][]{anaya2016biomedical, benkebir2019multi}} or the heuristics are irrelevant to the discussion \citep[e.g.,][which focuses on a comparison of single-trip and multiple-trip insertion heuristic, and is only applicable for the routing phase]{karoonsoontawong2015efficient}.


\begin{landscape}
\begin{table}[]
\scriptsize
    \centering
    \begin{tabular}{@{}
    >{\raggedright}p{1.5cm}                         % paper
    >{\raggedright}p{2.2cm}                         % method
    %>{\centering}p{1.3cm}              % solution construction
    >{\raggedright}p{2cm}                         % heuristic
    >{\raggedright}p{2cm}                         % customer operation
    >{\raggedright}p{2cm}                         % trip operation
    >{\centering\arraybackslash}p{1cm}            % infeasible intmd. sol.
    p{7.6cm}@{}}                                    % algorithm summary
    \toprule
        %& & \multicolumn{2}{c}{Routing Phase}
        %& \multicolumn{2}{c}{Local Search Moves} & & \\
        %\cmidrule{3-5}
            %& %& Feasible Sol.  
            & & Routing Phase & Customer & Trip
            & \multicolumn{1}{l}{Infeasible}
            & \multicolumn{1}{c}{Algorithm} \\
            Paper   &   Method  %&   Construction
            & Method  &   Operation     &   Operation
            &  \multicolumn{1}{l}{Intmd. Sol.}
            &  \multicolumn{1}{c}{Summary} \\
         \midrule
         \cite{battarra2009adaptive} 
         & Iterative Routing-Packing (IRP)
         %& N  
         & Sequential insertion, pivoted customers, parallel insertion & 2-opt  & - & N &
         IRP: two-step routing and greedy packing. An adaptive guidance mechanism influences the trips produced in routing heuristics: critical time intervals and critical commodities. \\ 
         \midrule
         \cite{wang2014metaheuristic}   
         & Adaptive Memory Procedure (AMP)
         %& N   
         & Optimal splitting procedure & Relocate, swap   & Trip partition   & N  & Construct trips based on optimal splitting procedure; assign trips in the pool to a vehicle, then perform LS. Iteratively update the trip pool with the best solution found.  \\  
         \midrule
         \cite{azi2014adaptive}
         & Adaptive Large Neighborhood Search (ALNS)
         %& Y  
         & Parallel construction (of journeys) & Remove related (spatial-temporal) customers & Remove related (spatially proximate) trips     & N
         & ALNS chooses destruction operators iteratively; chooses reconstruction operators probabilistically (weighted based on historical performance). Destruction operator is insertion with least-cost or regret-based heuristics. The acceptance criterion is similar to SA. \\
         \midrule
         \cite{cattaruzza2014iterated}
         & Iterated Local Search (ILS)
         %& Y 
         & Giant tour AdSplit  & Relocate, swap, 2-opt, 2-opt*  & Relocate, swap (trips)     & Y & ILS uses crossover operator on giant tour as a perturbation, AdSplit transforms such perturbation to a solution, LS improve the solution. If the final solution is infeasible, apply a repair procedure. The AdSplit is based on label generation, with a heuristic to limit the number of permutations.  \\ 
         \midrule
         \cite{cattaruzza2016multi}
         & Memetic Algorithm (MA)
         %& Y    
         & Initial population & Relocate, exchange, 2-opt* & Relocate, exchange (trips) & Y    
         & MA selects chromosomes with a binary tournament (crossover operator). Chromosomes are giant tours (i.e., permutations of nodes). LS moves to improve chromosomes with adjusted penalization. The AdSplit transforms chromosomes into solution.  \\
         \midrule
             \cite{despaux2016multi}    
             & Simulated Annealing (SA)
             %& Y
             & Solomon insertion heuristic                 
             & Revert order, relocation, swap
             & Relocate, divide overloaded trip, overflow in planning
             & Y
             & SA selects LS move uniformly; if the solution is infeasible, prioritize special moves. \\
         \midrule
         %\cite{benkebir2019multi}
         %& Hybrid Genetic Algorithm (HGA)
         %& Y
         %& Giant tour, Split, Graph-coloring                                        & 5 special LS moves                                                           & -                                                                      & Y                                                                                      & HGA has similar idea to MA in Cattaruzza et al. (2016)       
         %\\
         %\midrule
         \cite{franccois2019adaptive}
         & ALNS
         %& Y 
         & Sequential insertion                                                     & Relocate, exchange                                                           & -                                                                      & Y                                                                                      & ALNS with Multitrip Operators and ALNS combined with Bin Packing.         \\ 
         \midrule
         \cite{zhen2020multi}
         & Hybrid Particle Swarm Optimization (HPSO) / Hybrid Genetic Algorithm (HGA)
         %& N
         & Random
         & Relocate, exchange, revert order
         & -
         & Y
         & The HPSO extends PSO with a local search-variable neighborhood descent (LS-VND) and a solution translation mechanism, while the HGA incorporates a reorder routine to speed up the algorithm.
         \\ 
         \midrule
         \cite{pan2021multi}
         & ALNS
         %& Y
         & Regret insert                                                            & Relocate, swap                                                               & Remove 1 or 2 trips*                                                   & N                                                                                      & ALNS chooses destruction  reconstruction probabilistically (weighted based on historical performance). Reconstruction is insertion with regret insert; ALNS perturbation strength is increased gradually.
         \\
         \bottomrule
    \end{tabular}
    \caption{Heuristic Algorithms}
    \label{table:3}
\end{table}
\end{landscape}


%Difficult to evaluate the time window violation when the loading time is trip-dependent (cannot use Solomon's push-forward directly) (cite \cite{cattaruzza2016multi})

\section{Conclusions and open research areas}
\label{sec:trends}

This research is the first to review and classify the MTVRPTW literature.  In this study, we develop path-based flow formulations to introduce the MTVRPTW and its variants.  Then, we offer a taxonomic review of the MTVRPTW literature, including applications, mathematical models, and solution algorithms.  \add{This section also offers our perspective on current research trends and identifies areas of potential research focus.}
\newline

%In this section, we analyze trends in the MTVRPTW literature concerning the following aspects: (TODO: change this) choice of the MTVRPTW feature (i.e., which features are best suited for specific applications); exact and heuristic algorithm designs.  We also propose possible research areas, where applicable.
%\newline

% TODO: why some variants are more important (practically)? What is their practical importance?
% \newline

\add{Recent studies on the MTVRPTW have focused on real-life problems that involve various real-world aspects.  These complex real-life MTVRPTWs involve more complex constraints and objectives.  That is, real-life and rich MTVRPTWs appear to be the trend of the research on MTVRPTW.  According to \cite{lahyani2015rich}, a Rich Vehicle Routing Problem (RVRP) must extend the standard VRP by considering not less than four strategic or tactical characteristics regarding the distribution system and containing no less than six daily restrictions related to the physical characteristics.  In addition, if a VRP is mostly defined by strategic and tactical aspects (resp. by physical characteristics), at least five of them (resp. at least nine of them) must appear in an RVRP.  Based on their definition of RVRPs, some variants of the MTVRPTW in the literature may not be considered as rich.  Nonetheless, these previous studies seek to consider complex aspects of reality and show the trend toward real-life MTVRPTWs.}
\newline

% To capture various aspects of real-world vehicle routing problems, studies in the literature usually consider multiple features simultaneously (i.e., rich MTVRPTW variants).  For example, we have found that the following features are frequently grouped together: profit (P), service-dependent loading time (SDLT), and limited trip duration (LTD).  These features are particularly popular in studies with exact solution algorithms.  Moreover, the majority of exact algorithms in the literature rely on the enumeration of non-dominated paths, which is not suitable for MTVRPTW variants with loose deadline constraints (i.e., trip duration).  Furthermore, multiple mathematical formulations exist for the same MTVRPTW variant (e.g., MTVRPTPW-SDLT-LTD).  In our opinion, determining the most effective formulation for popular MTVRPTW variants deserves consideration in future research.  \add{We also note that there are real-world constraints that are not well-studied in the MTVRPTW literature.  For instance, \cite{anaya2016biomedical} consider multiple pick-up and delivery of perishable biomedical samples, which adds complexity to the formulation and solution.}

% We thus believe that research on new enumeration methods is needed to tackle such variants effectively.  Another potential direction is to extend the work in \cite{neira2020new}, which propose new MILP formulations capable of solving an MTVRPTW variant without trip duration limit (i.e., MTVRPTW-SDLT).

% citation for papers require non-dominated: (e.g., Azi et al., 2010 \cite{azi2010exact}; Macedo et al., 2011 \cite{macedo2011solving}; Macedo et al., 2012 \cite{macedo2012generalized}; Hernandez et al., 2014 \cite{hernandez2014new}; Hernandez et al., 2016 \cite{hernandez2016branch}; Paradiso et al., 2020 \cite{paradiso2020exact})

% For heuristic algorithms, the initial step of each algorithm can be generalized as a routing phase, which produces either a feasible solution or a candidate set of trips.  This first phase is relatively independent of the following phase(s), meaning that we can plug in different procedures for routing, but the algorithm still generates a final feasible solution.  However, we believe there is a lack of focus in the literature on analyzing the impact of the routing phase on the overall MTVRPTW algorithm's performance.  In other words, whether we should use a specific heuristic (in the routing phase) suitable for the corresponding metaheuristic or not (i.e., different routing heuristics have minimal impact on the final result) is an open question worth exploring in future research.  

\add{Concerning solution algorithms, as indicated by \cite{lahyani2015rich}, exact algorithms are rarely proposed to solve RVRPs because of their limited capability to tackle large-scale instances of such complex problems.  Likewise, exact algorithms are not, at least currently, promising for tackling various large-scale MTVRPTWs.  Indeed, existing studies of exact solution algorithms in the MTVRPTW literature are primarily confined to a few features: Profit, Dependent Loading or Unloading Times, and Limited Trip Duration.  In general, the most popular methods for solving real-life and rich MTVRPTWs are metaheuristics, namely: ILS, ALNS, SA, Memetic Algorithm, and adaptive memory procedure (AMP).  We note that metaheuristics in the MTVRPTW literature usually incorporate procedures customized for specific variants, which complicates the comparative study of different metaheuristics for MTVRPTWs.  Furthermore, hybrid solution techniques (i.e., matheuristics) that combine metaheuristics and exact methods have been successfully utilized to tackle large instances of complex RVRPs within a reasonable time \citep{doerner2010survey, goel2020hybrid}.  Thus, we believe that matheuristics are promising approaches for solving large-scale MTVRPTWs.}
\newline

% While it is possible to adapt an algorithm designed for one variant to solve another, we find such benchmarking method challenging as authors often include procedures customized for specific variants within their metaheuristics, leading to an unbalanced comparison. 
% Regardless of the choice of metaheuristic, we found that it is common to incorporate both time window features, such as critical time intervals, and LS moves tailored to multi-trip in MTVRPTW heuristic algorithms.

% Various heuristics (or metaheuristics) have been proposed to solve the MTVRPTW and its variants: ILS, ALNS, SA, Memetic Algorithm, and adaptive memory procedure (AMP).  A balance between diversification (i.e., exploration of new solution elements) and intensification (i.e., exploitation of champion features) is essential for efficient metaheuristics \citep{vidal2013heuristics}, albeit often paid little attention in the MTVRPTW literature.  Therefore, we believe that parameter analysis (i.e., tuning) for MTVRPTW solution algorithms is a promising future research area.  

% Subsequent research can also focus on conducting comparative studies of various heuristic algorithms for the MTVRPTW.  While it is possible to adapt an algorithm designed for one variant to solve another, we find such benchmarking method challenging as authors often include procedures customized for specific variants within their metaheuristics, leading to an unbalanced comparison.  Regardless of the choice of metaheuristic, we have found that it is common to incorporate both time window features, such as critical time intervals, and LS moves tailored to multi-trip in MTVRPTW heuristic algorithms.
% \newline

% Regardless of the choice of metaheuristic, we believe that emphases on both time window features, such as critical time intervals, and LS moves tailored to multi-trip, are essential in designing state-of-the-art heuristic algorithms for the MTVRPTW.

\add{Beyond the aforementioned methods, \textit{Reinforcement Learning} (RL) techniques also show potential in tackling MTVRPTWs, building on their successful application in solving VRPs.  For example, \cite{zhang2020multi} propose a multi-agent RL model for the VRP with soft time windows that outperforms classical heuristics in terms of computation time; \cite{lu2020learning} combine RL and classical heuristics to address the VRP by using RL-based controller to select the improvement operator in an iterative algorithm.  \cite{raza2022vehicle}, which survey recent advancements in solving VRPs using RL, note that there is a lack of RL studies focusing on real-life constraints.  To our knowledge, RL studies targeting the MTVRPTW and its variants have not yet been proposed.  Moreover, as RL models that work well for a problem may perform inferior for other variants \citep[see,][]{li2021deep}, we believe that it is worth researching RL frameworks with a high level of flexibility to solve different MTVRPTW variants effectively.}

% Apart from the heuristics mentioned above, \textit{Reinforcement Learning} (RL) techniques have been applied to solve VRPs \citep[see,][]{raza2022vehicle}.  \cite{raza2022vehicle} also point out a lack of RL studies focusing on real-life VRP constraints.  To our knowledge, RL models designed for both the time windows and multi-trip characteristics of the VRP (i.e., the MTVRPTW), have not yet been proposed.  Nevertheless, deep RL and multi-agent RL have shown their potential to tackle the VRP with time windows \citep[see,][]{gupta2022deep}, the VRP with soft time windows \citep[see,][]{zhang2020multi}, and the VRP with heterogeneous fleet \citep[see,][]{li2021deep}
%https://pubsonline.informs.org/doi/abs/10.1287/inte.2021.1108, also check \cite{raza2022vehicle}).  

% Although generalized RL frameworks exist for the general planning problems \citep[see,][]{groshev2018learning}, problem-specific policies utilizing both time window and multi-trip characteristics of the MTVRPTW can yield better results in terms of both solution quality and runtime.

% However, we found a lack of RL studies focusing on real-life MTVRPTWs.  Furthermore, a major limitation of RL algorithms is the need for re-training for new problem variants \citep{raza2022vehicle}.

%Thus, we believe applying RL to handle real-life and rich MTVRPTWs can be a promising direction for future study.
% However, \cite{raza2022vehicle} also point out a lack of RL studies focusing on real-life VRP constraints.

% Apart from the methods mentioned above, reinforcement learning has shown its potential to tackle hard combinatorial optimization problems.  \cite{bello2016neural} propose a framework based on neural networks and reinforcement learning, which can find near-optimal solutions for the Travelling Salesman Problem instances with up to 100 customers.  For the VRP, \cite{nazari2018reinforcement} present a reinforcement learning framework that outperforms classical heuristics on medium-sized instances.  Recently, \cite{gupta2022deep} suggest a deep Q-network method to solve large-scale instances of the VRPTW.  To our knowledge, reinforcement learning and machine learning, in general, have not yet been used to solve the MTVRPTW.  Hence, applying machine learning to handle the MTVRPTW and its variants can be a promising direction for future study.


%TODO: which variant lacks attention in the literature (for exact / heuristic method); which feature is not being directly attacked (like algorithms not designed with that feature in mind) -- "due to the large number of constraints in MTVRPTW, some features like abc xyz are often left un-attacked (i.e., algorithms not tailored for such feature)"

%\section{Concluding remarks}
%\label{sec:conclusion}

% The MTVRPTW is ...

%To the best of our knowledge, this research is the first to review and classify the MTVRPTW literature.  In this study, we develop path-based flow formulations to introduce the MTVRPTW and its variants.  Then, we present a taxonomic review of the MTVRPTW literature based on applications, mathematical models, and solution algorithms.  Lastly, we offer our perspective on current research trends and identify areas of potential research focus.
%\newline

% \begin{table}[]
% \small
%     \centering
%     %\begin{tabular}{@{}lc>{\raggedright}p{2cm}>{\raggedright}p{1.4cm}>{\raggedright}p{1.4cm}cp{5cm}@{}}
%     \begin{tabular}{@{}>{\raggedright}p{1.8cm}>{\raggedright}p{3cm}>{\raggedright}p{4.5cm}l@{}}
%     \toprule
%          Paper  &   Method  &   Model   &   Column representation \\
%          \midrule
%          Azi et al. (2010) \cite{azi2010exact}
%          & Branch-and-price
%          & Master problem: set packing
%             \newline Pricing subproblem: elementary shortest-path problem with resource constraints (ESPPRC)
%          & Journey \\
%          \midrule
%          Macedo et al. (2011) \cite{macedo2011solving}
%          & Network flow, time discretization
%          & Minimum flow problem
%             \newline Solution: paths in network flow
%             \newline Nodes: discrete time instants
%          & - \\
%          \midrule
%          Macedo et al. (2012) \cite{macedo2012generalized}
%          & Network flow, time discretization
%          & Minimum flow problem
%             \newline Solution: paths in network flow
%             \newline Nodes: discrete time instants
%          & - \\
%          \midrule
%          Hernandez et al. (2014) \cite{hernandez2014new}
%          & Branch-and-price
%          & Master problem: set covering
%             \newline Pricing subproblem: select timing for trips
%          & Trip \\
%          \midrule
%          Hernandez et al. (2016) \cite{hernandez2016branch}
%          & Branch-and-price
%          & Master problem: set covering
%             \newline Pricing subproblem: ESPPRC
%          & Journey / Trip \\
%          \midrule
%          Paradiso et al. (2020) \cite{paradiso2020exact}
%          & Branch-and-cut with embeded column generation
%          & (Separation problem of branch-and-cut) Team Orienteering Problem with Time Windows
%          & Structure \\
%          \midrule
%          Neira et al. (2020) \cite{neira2020new}
%          & MILP formulations implemented directly in CPLEX
%          & -
%          & - \\
%          \midrule
%          Huang et al. (2021) \cite{huang2021multi}
%          & Branch-and-price
%          & Master problem: set partitioning
%             \newline Pricing subproblem: ESPPRC
%          & Trip \\
%          \midrule
%          Yang (2022) \cite{yang2023exact}
%          & Price-cut-and-enumerate
%          & (Separation problem of branch-and-cut) Team Orienteering Problem with Time Windows
%          & Structure \\
%          \bottomrule
%     \end{tabular}
%     \caption{Exact Algorithms}
%     \label{table:2}
% \end{table}


\clearpage

\bibliography{refs} % Entries are in the refs.bib file

\end{document}


